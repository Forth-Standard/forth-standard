\annex{References Implementations} % I. (informative annex)
\label{annex:implement}

\section{Introduction} % I.1

\marginpar{\tiny ed09}\cbstart
In the most recent review of this document, proposals were encouraged
to include a references implementation where possible.  Where an
implementation requires system specific knowledge it was documented.


This appendix contains the reference implementations that have been
accepted by the committee.  This is not a complete reference
implementation nor do the committee recommend these implementations.
They are provided solely for the purpose of providing a detailed
understanding of a definitions requirement.  A systems implementor is
free to implement any operation in a manner that suites their system,
but it must exhibit the same behavior as the reference implementation
given here.

This appendix contains the reference implementations that were
accepted by the committee.  It is not complete and it is not the
committeeies intention to recommend these implementations.
The code is provided to give a systems implementor an indication as
to the operation of the word, or set of words, given in the reference
implemention.  A systems implementaor is free to implement any
operation in a manner that suites their system, but which exhibits
the same behavior as the reference implementation.
\cbend


\ifinline
	\newcommand{\impsection}[2]{%
		\section{The optional #2 word set}
		\fbox{\parbox{\linewidth}{\slshape
		In the \emph{review} (r) version of the document the
		reference implementation for a word is given in the main
		definition of the word.  The implementation of words in
		the \textbf{#1} word set will appear here in the final
		document.
		}}
	}
\else
	\namespace{imp}
	\newcommand{\impsection}[2]{%
		\defersection{#2}
		\setwordlist{#1}
		\input{i-#1.sub}
		\stepsection
	}
	\defersection{}
\fi

\setcounter{section}{5}
\section{The Core word set}				% I.6
\ifinline
	\fbox{\parbox{\linewidth}{\slshape
	In the \emph{review} (r) version of the document the
	reference implementaiton for a word are given in the main
	defintion of the word.  The implementation of words
	in the \textbf{core} word set will appear here in the
	final document.
	}}
\else
	\input{i-core.sub}
\fi

\impsection{block}{Block}				% I.7
\impsection{double}{Double-Number}		% I.8
\impsection{exception}{Exception}		% I.9
\impsection{facility}{Facility}			% I.10
\impsection{file}{File-Access}			% I.11
\impsection{floating}{Floating-Point}	% I.12
\impsection{local}{Locals}				% I.13
\impsection{memory}{Memory-Allocation}	% I.14
\impsection{tools}{Programming-Tools}	% I.15
\impsection{search}{Search-Order}		% I.16
\impsection{string}{String}				% I.17
