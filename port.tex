\annex{ANS Forth portability guide} % E.  (informative annex)
\label{annex:port}

\section{Introduction} % E.1

The most popular architectures used to implement Forth have had
byte-addressed memory, 16-bit op\-er\-a\-tions, and two's-complement
number representation. The Forth-83 Standard dictates that these
particular features must be present in a Forth-83 Standard system
and that Forth-83 programs may exploit these features freely.

However, there are many beasts in the architectural jungle that are
bit addressed or cell addressed, or prefer 32-bit operations, or
represent numbers in one's complement. Since one of Forth's strengths
is its usefulness in ``strange'' environments on ``unusual'' hardware
with ``peculiar'' features, it is important that a Standard Forth run
on these machines too.

A primary goal of the ANS Forth Standard is to increase the types of
machines that can support a Standard Forth. This is accomplished by
allowing some key Forth terms to be implementation-defined (e.g., how
big is a cell?) and by providing Forth operators (words) that conceal
the implementation. This frees the implementor to produce the Forth
system that most effectively utilizes the native hardware. The machine
independent operators, together with some programmer discipline, enable
a programmer to write Forth programs that work on a wide variety of
machines.

The remainder of this Annex provides guidelines for writing portable
ANS Forth programs. The first section describes ways to make a program
hardware independent. It is difficult for someone familiar with only
one machine architecture to imagine the problems caused by transporting
programs between dissimilar machines. Consequently, examples of specific
architectures with their respective problems are given. The second
section describes assumptions about Forth implementations that many
programmers make, but can't be relied upon in a portable program.

\section{Hardware peculiarities} % E.2
\label{port:hardware}

\subsection{Data/memory abstraction} % E.2.1

Data and memory are the stones and mortar of program construction.
Unfortunately, each computer treats data and memory differently. The
ANS Forth Systems Standard gives definitions of data and memory that
apply to a wide variety of computers. These definitions give us a way
to talk about the common elements of data and memory while ignoring
the details of specific hardware. Similarly, ANS Forth programs that
use data and memory in ways that conform to these definitions can
also ignore hardware details. The following sections discuss the
definitions and describe how to write programs that are independent
of the data/memory peculiarities of different computers.

\subsection{Definitions} % E.2.2

Three terms defined by ANS Forth are address unit, cell, and character.
The address space of an ANS Forth system is divided into an array of
address units; an address unit is the smallest collection of bits that
can be addressed. In other words, an address unit is the number of
bits spanned by the addresses \emph{addr} and \emph{addr}+1. The most
prevalent machines use 8-bit address units. Such ``byte addressed''
machines include the Intel 8086 and Motorola 68000 families. However,
other address unit sizes exist. There are machines that are bit
addressed and machines that are 4-bit nibble addressed. There are
also machines with address units larger than 8-bits. For example,
several Forth-in-hardware computers are cell addressed.

The cell is the fundamental data type of a Forth system. A cell can
be a single-cell integer or a memory address. Forth's parameter and
return stacks are stacks of cells. Forth 83 specifies that a cell is
16-bits. In ANS Forth the size of a cell is an implementation-defined
number of address units. Thus, an ANS Forth implemented on a 16-bit
microprocessor could use a 16-bit cell and an implementation on a
32-bit machine could use a 32-bit cell. Also 18-bit machines, 36-bit
machines, etc., could support ANS Forth systems with 18 or 36-bit
cells respectively. In all of these systems, \word{DUP} does the same
thing: it duplicates the top of the data stack. \word{!} (store)
behaves consistently too: given two cells on the data stack it
stores the second cell in the memory location designated by the
top cell.

Similarly, the definition of a character has been generalized to be
an implementation-defined number of address units (but at least eight
bits). This removes the need for a Forth implementor to provide 8-bit
characters on processors where it is inappropriate. For example, on
an 18-bit machine with a 9-bit address unit, a 9-bit character would
be most convenient. Since, by definition, you can't address anything
smaller than an address unit, a character must be at least as big as
an address unit. This will result in big characters on machines with
large address units. An example is a 16-bit cell addressed machine
where a 16-bit character makes the most sense.

\subsection{Addressing memory} % E.2.3

ANS Forth eliminates many portability problems by using the above
definitions. One of the most common portability problems is addressing
successive cells in memory. Given the memory address of a cell, how
do you find the address of the next cell? In Forth 83 this is easy:
\texttt{2 +}. This code assumes that memory is addressed in 8-bit
units (bytes) and a cell is 16-bits wide. On a byte-addressed machine
with 32-bit cells the code to find the next cell would be \texttt{4 +}.
The code would be \word{1+} on a cell-addressed processor and
\texttt{16 +} on a bit-addressed processor with 16-bit cells. ANS Forth
provides a next-cell operator named \word{CELL+} that can be used in
all of these cases. Given an address, \word{CELL+} adjusts the address
by the size of a cell (measured in address units). A related problem
is that of addressing an array of cells in an arbitrary order. A
defining word to create an array of cells using Forth 83 would be:
\begin{quote}\ttfamily
	\word{:} ARRAY ~ \word{CREATE} ~ \word{2*} \word{ALLOT} ~
		\word{DOES} \word{SWAP} \word{2*} \word{+} \word{;}
\end{quote}
Use of \word{2*} to scale the array index assumes byte addressing and
16-bit cells again. As in the example above, different versions of
the code would be needed for different machines. ANS Forth provides
a portable scaling operator named \word{CELLS}. Given a number \param{n},
\word{CELLS} returns the number of address units needed to hold \param{n}
cells. A portable definition of array is:
\begin{quote}\ttfamily
	\word{:} ARRAY ~ \word{CREATE} ~ \word{CELLS} \word{ALLOT} \\
	\hspace*{2em}\word{DOES} \word{SWAP} \word{CELLS} \word{+} \word{;}
\end{quote}
There are also portability problems with addressing arrays of
characters. In Forth 83 (and in the most common ANS Forth
implementations), the size of a character will equal the size of an
address unit. Consequently addresses of successive characters in
memory can be found using 1+ and scaling indices into a character
array is a no-op (i.e., \texttt{1 *}). However, there are cases where
a character is larger than an address unit. Examples include (1)
systems with small address units (e.g., bit- and nibble-addressed
systems), and (2) systems with large character sets (e.g., 16-bit
characters on a byte-addressed machine). \word{CHAR+} and \word{CHARS}
operators, analogous to \word{CELL+} and \word{CELLS} are available
to allow maximum portability.

ANS Forth generalizes the definition of some Forth words that operate
on chunks of memory to use address units. One example is \word{ALLOT}.
By prefixing \word{ALLOT} with the appropriate scaling operator
(\word{CELLS}, \word{CHARS}, etc.), space for any desired data structure
can be allocated (see definition of array above). For example:
\begin{quote}\ttfamily
	\word{CREATE} ABUFFER 5 \word{CHARS} \word{ALLOT}
	\word{p} allot 5 character buffer)
\end{quote}
The memory-block-move word also uses address units:
\begin{quote}\ttfamily
source destination 8 \word{CELLS} \word{MOVE} \word{p} move 8 cells)
\end{quote}


\subsection{Alignment problems} % E.2.4

Not all addresses are created equal. Many processors have restrictions
on the addresses that can be used by memory access instructions. This
Standard does not require an implementor of an ANS Forth to make
alignment transparent; on the contrary, it requires (in Section
\xref[3.3.3.1 Address alignment]{usage:aaddr}) that an ANS Forth
program assume that character and cell alignment may be required.

 One of the most common problems caused by alignment restrictions
 is in creating tables containing both characters and cells. When
 \word{,} (comma) or \word{C,} is used to initialize a table, data
 is stored at the data-space pointer. Consequently, it must be
 suitably aligned. For example, a non-portable table definition
 would be:
\begin{quote}\ttfamily
	\word{CREATE} ATABLE 1 \word{C,} X \word{,} 2 \word{C,} Y \word{,}
\end{quote}
On a machine that restricts 16-bit fetches to even addresses,
\word{CREATE} would leave the data space pointer at an even address,
the 1 \word{C,} would make the data space pointer odd, and \word{,}
(comma) would violate the address restriction by storing \texttt{X}
at an odd address. A portable way to create the table is:
\begin{quote}\ttfamily
	\word{CREATE} ATABLE 1 \word{C,}
		\word{ALIGN} X \word{,} 2 \word{C,} \word{ALIGN} Y \word{,}
\end{quote}
\word{ALIGN} adjusts the data space pointer to the first aligned
address greater than or equal to its current address. An aligned
address is suitable for storing or fetching characters, cells, cell
pairs, or double-cell numbers.

After initializing the table, we would also like to read values from
the table. For example, assume we want to fetch the first cell,
\texttt{X}, from the table. \texttt{ATABLE} \word{CHAR+} gives the
address of the first thing after the character. However this may not
be the address of \texttt{X} since we aligned the dictionary pointer
between the \word{C,} and the \word{,}. The portable way to get the
address of \texttt{X} is:
\begin{quote}\ttfamily
	ATABLE \word{CHAR+} \word{ALIGNED}
\end{quote}
\word{ALIGNED} adjusts the address on top of the stack to the first
aligned address greater than or equal to its current value.


\section{Number representation} % E.3

Different computers represent numbers in different ways. An awareness
of these differences can help a programmer avoid writing a program
that depends on a particular representation.

\subsection{Big endian vs. little endian} % E.3.1

The constituent bits of a number in memory are kept in different
orders on different machines. Some machines place the most-significant
part of a number at an address in memory with less-significant parts
following it at higher addresses. Other machines do the opposite ---
the least-significant part is stored at the lowest address. For
example, the following code for a 16-bit 8086 ``little endian'' Forth
would produce the answer 34 (hex):
\begin{quote}\ttfamily
	\word{VARIABLE} FOO
	\quad \word{HEX} 1234 FOO \word{!}
	\quad FOO \word{C@}
\end{quote}
The same code on a 16-bit 68000 ``big endian'' Forth would produce the
answer 12 (hex). A portable program cannot exploit the representation
of a number in memory.

A related issue is the representation of cell pairs and double-cell
numbers in memory. When a cell pair is moved from the stack to memory
with \word{2!}, the cell that was on top of the stack is placed at the
lower memory address. It is useful and reasonable to manipulate the
individual cells when they are in memory.

\subsection{ALU organization} % E.3.2

Different computers use different bit patterns to represent integers.
Possibilities include binary representations (two's complement, one's
complement, sign magnitude, etc.) and decimal representations (BCD,
etc.). Each of these formats creates advantages and disadvantages in
the design of a computer's arithmetic logic unit (ALU). The most
commonly used representation, two's complement, is popular because of
the simplicity of its addition and subtraction algorithms.

Programmers who have grown up on two's complement machines tend to
become intimate with their representation of numbers and take some
properties of that representation for granted. For example, a trick
to find the remainder of a number divided by a power of two is to mask
off some bits with \word{AND}. A common application of this trick is
to test a number for oddness using 1 \word{AND}. However, this will
not work on a one's complement machine if the number is negative (a
portable technique is 2 \word{MOD}).

The remainder of this section is a (non-exhaustive) list of things to
watch for when portability between machines with binary representations
other than two's complement is desired.

To convert a single-cell number to a double-cell number, ANS Forth
provides the operator \word{StoD}. To convert a double-cell number to
single-cell, Forth programmers have traditionally used \word{DROP}.
However, this trick doesn't work on sign-magnitude machines. For
portability a \word[double]{DtoS} operator is available. Converting an
unsigned single-cell number to a double-cell number can be done portably
by pushing a zero on the stack.


\section{Forth system implementation} % E.4

During Forth's history, an amazing variety of implementation techniques
have been developed. The ANS Forth Standard encourages this diversity
and consequently restricts the assumptions a user can make about the
underlying implementation of an ANS Forth system. Users of a particular
Forth implementation frequently become accustomed to aspects of the
implementation and assume they are common to all Forths. This section
points out many of these incorrect assumptions.

\subsection{Definitions} % E.4.1

Traditionally, Forth definitions have consisted of the name of the
Forth word, a dictionary search link, data describing how to execute
the definition, and parameters describing the definition itself. These
components are called the name, link, code, and parameter fields%
\footnote{These terms are not defined in the Standard.
They are mentioned here for historical continuity.
}. No method for accessing these fields has been found that works
across all of the Forth implementations currently in use. Therefore,
ANS Forth severely restricts how the fields may be used. Specifically,
a portable ANS Forth program may not use the name, link, or code field
in any way. Use of the parameter field (renamed to data field for
clarity) is limited to the operations described below.

Only words defined with \word{CREATE} or with other defining words
that call \word{CREATE} have data fields. The other defining words
in the Standard (\word{VARIABLE}, \word{CONSTANT}, \word{:}, etc.)
might not be implemented with \word{CREATE}. Consequently, a Standard
Program must assume that words defined by \word{VARIABLE},
\word{CONSTANT}, \word{:}, etc., may have no data fields. There is no
way for a Standard Program to modify the value of a constant or to
change the meaning of a colon definition. The \word{DOES} part of a
defining word operates on a data field. Since only \word{CREATE}d words
have data fields, \word{DOES} can only be paired with \word{CREATE} or
words that call \word{CREATE}.

In ANS Forth, \word{FIND}, \word{[']} and \word{'} (tick) return an
unspecified entity called an ``execution token''. There are only a
few things that may be done with an execution token. The token may be
passed to \word{EXECUTE} to execute the word ticked or compiled into
the current definition with \word{COMPILE,}. The token can also be
stored in a variable and used later. Finally, if the word ticked was
defined via \word{CREATE}, \word{toBODY} converts the execution token
into the word's data-field address.

One thing that definitely cannot be done with an execution token is
use \word{!} or \word{,} to store it into the object code of a Forth
definition. This technique is sometimes used in implementations where
the object code is a list of addresses (threaded code) and an execution
token is also an address. However, ANS Forth permits native code
implementations where this will not work.


\subsection{Stacks} % E.4.2

In some Forth implementations, it is possible to find the address of
a stack in memory and manipulate the stack as an array of cells. This
technique is not portable, however. On some systems, especially
Forth-in-hardware systems, the stacks might be in a part of memory
that can't be addressed by the program or might not be in memory at
all. Forth's parameter and return stacks must be treated as stacks.

A Standard Program may use the return stack directly only for
temporarily storing values. Every value examined or removed from the
return stack using \word{R@}, \word{Rfrom}, or \word{2Rfrom} must have been
put on the stack explicitly using \word{toR} or \word{2toR}. Even this
must be done carefully since the system may use the return stack to
hold return addresses and loop-control parameters. Section
\xref[3.2.3.3 Return stack]{usage:returnstack} of the Standard has a
list of restrictions.


\section{ROMed application disciplines and conventions} % E.5

When a Standard System provides a data space which is uniformly
readable and writeable we may term this environment ``RAM-only''.

Programs designed for ROMed application must divide data space into
at least two parts: a writeable and readable uninitialized part,
called ``RAM'', and a read-only initialized part, called ``ROM''.
A third possibility, a writeable and readable initialized part,
normally called ``initialized RAM'', is not addressed by this
discipline. A Standard Program must explicitly initialize the RAM
data space as needed.

The separation of data space into RAM and ROM is meaningful only
during the generation of the ROMed program. If the ROMed program is
itself a standard development system, it has the same taxonomy as an
ordinary RAM-only system.

The words affected by conversion from a RAM-only to a mixed RAM and
ROM environment are:
\begin{quote}
	\word{,} (comma) ~ \word{ALIGN} ~ \word{ALIGNED} ~ \word{ALLOT} ~
	\word{C,} ~ \word{CREATE} ~ \word{HERE} ~ \word{UNUSED}

	(\word{VARIABLE} always accesses the RAM data space.)
\end{quote}
With the exception of \word{,} (comma) and \word{C,} these words are
meaningful in both RAM and ROM data space. To select the data space,
these words could be preceded by selectors \texttt{RAM} and
\texttt{ROM}. For example:
\begin{quote}\ttfamily
	ROM ~ \word{CREATE} ONES ~ 32 \word{ALLOT} ~ ONES 32 1 \word{FILL} ~ RAM
\end{quote}
would create a table of ones in the ROM data space. The storage of
data into RAM data space when generating a program for ROM would be
an ambiguous condition.

A straightforward implementation of these selectors would maintain
separate address counters for each space. A counter value would be
returned by \word{HERE} and altered by \word{,} (comma), \word{C,},
\word{ALIGN}, and \word{ALLOT}, with \texttt{RAM} and \texttt{ROM}
simply selecting the appropriate address counter. This technique
could be extended to additional partitions of the data space.


\section{Summary} % E.6

The ANS Forth Standard cannot and should not force anyone to write
a portable program. In situations where performance is paramount,
the programmer is encouraged to use every trick in the book. On the
other hand, if portability to a wide variety of systems is needed,
ANS Forth provides the tools to accomplish this. There is probably
no such thing as a completely portable program. A programmer, using
this guide, should intelligently weigh the tradeoffs of providing
portability to specific machines. For example, machines that use
sign-magnitude numbers are rare and probably don't deserve much
thought. But, systems with different cell sizes will certainly be
encountered and should be provided for. In general, making a program
portable clarifies both the programmer's thinking process and the
final program.
