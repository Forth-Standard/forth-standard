% !TeX root = forth.tex
% !TeX spell = en_US

\vspace*{-6ex}\chapter*{Foreword}
\addcontentsline{toc}{section}{Foreword}
\label{foreword}
\markboth{Foreword}{Foreword}

Forth is a language for direct communication between human beings and
machines. Forth was invented by Charles Moore to increase programmer
productivity without sacrificing machine efficiency.
Using natural-language diction and machine-oriented syntax,
Forth provides an economical, productive environment for interactive
compilation and execution of programs. Forth also provides low-level
access to computer-controlled hardware, and the ability to extend the
language itself. This extensibility allows the language to be quickly
expanded and adapted to special needs and different hardware systems.
Forth provides for highly interactive program development and testing.

In the interests of transportability of application software written in
Forth, standardization efforts began in the mid-1970s by an international
group of users and implementors who adopted the name ``Forth Standards Team''.
This effort resulted in the Forth-77 Standard. As the language continued
to evolve, an interim Forth-78 Standard was published by the Forth Standards
Team. Following Forth Standards Team meetings in 1979, the Forth-79 Standard
was published in 1980. Major changes were made by the Forth Standards Team
in the Forth-83 Standard, which was published in 1983.

The ANS Forth committee was formed in 1987 to address the fragmentation
within the Forth community caused not only by the difference between
Forth 79 and Forth 83 but the exploitation of technical de\-vel\-op\-ments.
Undertaking a comprehensive review of existing implementations they
moved away from prescribing stringent requirements, preferring to
describe the operation of the virtual machine, without reference to
an implementation.  The ANS Forth Standard was published in
1994\footnote{ANSI X3.215--1994 Information Systems --- Programming Language FORTH}
and was adopted as an international standard in
1997\footnote{ISO/IEC 15145:1997 Information technology.  Programming languages.  FORTH}.

The Forth 200\emph{x} Standardisation Committee was formed in 2004
to allow the Forth community to contribute to an updated standard.
Changes are proposed and discussed in the electronic media:
the usenet news group \texttt{comp.lang.forth};
the \texttt{forth200x@yahoogroups.com} email list;
the \texttt{www.forth200x.org} web site.
Annual public meeting are held to review and vote on the proposed
changes.

This document is the result of the public review meetings first held on
October 21--22, 2005 (Santander) and subsequently on
September 14--15, 2006 (Cambridge),
September 13--14, 2007 (Dagstuhl),
September 25--26, 2008 (Vienna),
March 25--27, 2009 (Neuenkirchen, Rheine),
September 2--4, 2009 (Exeter),
March 24--26, 2010 (Rostock),
September 22--24, 2010 (Hamburg),
September 21--23, 2011 (Vienna),
September 12--14, 2012 (Oxford),
September 25--27, 2013 (Ham\-burg).
