% !TeX root = forth.tex
% !TeX spellcheck = en_US
% !TeX program = pdflatex

\chapter{The optional Search-Order word set} % 16
\wordlist{search}

\section{Introduction} % 16.1

\section{Additional terms and notation} % 16.2
\label{search:terms}

\begin{description}
\item[compilation word list:]
	The word list into which new definition names are placed.

\item[search order:]
	A list of word lists specifying the order in which the
	dictionary will be searched.

\item[find:]
	Update \xref{notation:terms} definition of find to search the search
	order or a specified wordlist fora definition name matching a given
	string.
\end{description}

\section{Additional usage requirements} % 16.3

\subsection{Data types} % 16.3.1

Word list identifiers are implementation-dependent single-cell
values that identify word lists.

Append table \ref{search:types} to table \ref{table:datatypes}.

\begin{table}[h]
  \begin{center}
	\caption{Data types}
	\label{search:types}
	\begin{tabular}{llr}
	\hline\hline
	\emph{Symbol} & \emph{Data type} & \emph{Size on stack} \\
	\hline
	\emph{wid}		& word list identifiers	& 1 cell \\
	\hline\hline
	\end{tabular}
  \end{center}
\end{table}

See: \xref[3.1 Data types]{usage:data},
\xref[3.4.2 Finding definition names]{usage:find},
\xref[3.4 The Forth text interpreter]{usage:command}.

\subsection{Environmental queries} % 16.3.2

Append table \ref{search:env} to table \ref{table:env}.

See: \xref[3.2.6 Environmental queries]{usage:env}.

\begin{table}[ht]
  \begin{center}
	\caption{Environmental Query Strings}
	\label{search:env}
	\begin{tabular}{p{10em}rcp{0.42\textwidth}}
		\hline\hline
		\multicolumn{2}{l}{String \hfill Value data type} & Constant? & Meaning \\
		\hline
		\texttt{WORDLISTS}			& \emph{n}		& yes	&
			maximum number of word lists usable in the search order\\
		\hline\hline
	\end{tabular}
  \end{center}
\end{table}

\subsection{Finding definition names} % 16.3.3
\label{search:find}

When searching a word list for a definition name, the system shall
search each word list from its last definition to its first. The
search may encompass only a single word list, as with
\word{SEARCH-WORDLIST}, or all the word lists in the search order,
as with the text interpreter and \word{FIND}.

Changing the search order shall only affect the subsequent finding
of definition names in the dictionary. A system with the Search-Order
word set shall allow at least eight word lists in the search order.

An ambiguous condition exists if a program changes the compilation
word list during the compilation of a definition or before
modification of the behavior of the most recently compiled definition
with \word[tools]{;CODE}, \word[core]{DOES}, or
\word[core]{IMMEDIATE}.

A program that requires more than eight word lists in the search
order has an environmental dependency.

See: \xref[3.4.2 Finding definition names]{usage:find}.


\subsection{Contiguous regions} % 16.3.4

The regions of data space produced by the operations described in
\xref[3.3.3.2 Contiguous regions]{usage:contiguous} may be
non-contiguous if \word{WORDLIST} is executed between allocations.

\section{Additional documentation requirements} % 16.4

\subsection{System documentation} % 16.4.1

\subsubsection{Implementation-defined options} % 16.4.1.1

\begin{itemize}
\item maximum number of word lists in the search order
	(\xref[16.3.3 Finding definition names]{search:find},
	 \wref{search:SET-ORDER}{SET-ORDER});
\item minimum search order
	(\wref{search:SET-ORDER}{SET-ORDER},
	 \wref{search:ONLY}{ONLY}).
\end{itemize}

\subsubsection{Ambiguous conditions} % 16.4.1.2

\begin{itemize}
\item changing the compilation word list
	(\xref[16.3.3 Finding definition names]{search:find});
\item search order empty
	(\wref{search:PREVIOUS}{PREVIOUS});
\item too many word lists in search order
	(\wref{search:ALSO}{ALSO}).
\end{itemize}

\subsubsection{Other system documentation} % 16.4.1.3

\begin{itemize}
\item no additional requirements.
\end{itemize}

\subsection{Program documentation} % 16.4.2

\subsubsection{Environmental dependencies} % 16.4.2.1

\begin{itemize}
\item requiring more than eight word-lists in the search order
	(\xref[16.3.3 Finding definition names]{search:find}).
\end{itemize}

\subsubsection{Other program documentation} % 16.4.2.2

\begin{itemize}
\item no additional requirements.
\end{itemize}


\section{Compliance and labeling} % 16.5

\subsection{Forth-\snapshot{} systems} % 16.5.1

The phrase ``Providing the Search-Order word set'' shall be
appended to the label of any Standard System that provides all of
the Search-Order word set.

The phrase ``Providing \emph{name(s)} from the Search-Order
Extensions word set'' shall be appended to the label of any
Standard System that provides portions of the Search-Order
Extensions word set.

The phrase ``Providing the Search-Order Extensions word set'' shall
be appended to the label of any Standard System that provides all of
the Search-Order and Search-Order Extensions word sets.

\subsection{Forth-\snapshot{} programs} % 16.5.2

The phrase ``Requiring the Search-Order word set'' shall be appended
to the label of Standard Programs that require the system to provide
the Search-Order word set.

The phrase ``Requiring \emph{name(s)} from the Search-Order
Extensions word set'' shall be appended to the label of Standard
Programs that require the system to provide portions of the
Search-Order Extensions word set.

The phrase ``Requiring the Search-Order Extensions word set'' shall
be appended to the label of Standard Programs that require the system
to provide all of the Search-Order and Search-Order Extensions word
sets.


\section{Glossary} % 16.6

\subsection{Search-Order words} % 16.6.1

\begin{worddef}{1180}{DEFINITIONS}
\item \stack{}{}

	Make the compilation word list the same as the first word list
	in the search order. Specifies that the names of subsequent
	definitions will be placed in the compilation word list.
	Subsequent changes in the search order will not affect the
	compilation word list.

\see \xref[16.3.3 Finding Definition Names]{search:find}.

	\begin{implement} % I.16.6.1.1180 DEFINITIONS
		\word{:} discard \word{p} x1 {\ldots} xn u -{}- ) \bs{} \textdf{Drop u+1 stack items} \\
		\tab 0 \word{qDO} \word{DROP} \word{LOOP} \\
		\word{;}

		\word{:} \word{DEFINITIONS} \word{p} -{}- ) \\
		\tab \word{GET-ORDER} \word{SWAP}  \word{SET-CURRENT} discard \\
		\word{;}
	\end{implement}

	\begin{testing} % T.16.6.1.1180 DEFINITIONS
		\ttfamily
		\test{\word{ONLY} \word{FORTH} \word{DEFINITIONS}}{} \\
		\test{\word{GET-CURRENT}}{FORTH-WORDLIST}

		\test{\word{GET-ORDER} wid2 \word{@} \word{SWAP} \word{1+} \word{SET-ORDER} \word{DEFINITIONS} \word{GET-CURRENT}\\}{wid2 \word{@}} \\
		\test{\word{GET-ORDER}}{get-orderlist wid2 \word{@} \word{SWAP} \word{1+}} \\
		\test{\word{PREVIOUS} \word{GET-ORDER}}{get-orderlist} \\
		\test{\word{DEFINITIONS} \word{GET-CURRENT}}{FORTH-WORDLIST}

		\word{:} alsowid2 \word{ALSO} \word{GET-ORDER} wid2 \word{@} \word{ROT} \word{DROP} \word{SWAP} \word{SET-ORDER} \word{;} \\
		alsowid2 \\
		\word{:} w1 1234 \word{;} \\
		\word{DEFINITIONS}
		\word{:} w1 -9876 \word{;} \word{IMMEDIATE}

		\word{ONLY} \word{FORTH} \\
		\test{w1}{1234} \\
		\word{DEFINITIONS} \\
		\test{w1}{1234} \\
		alsowid2 \\
		\test{w1}{-9876} \\
		\word{DEFINITIONS}
		\test{w1}{-9876}

		\word{ONLY} \word{FORTH} \word{DEFINITIONS} \\
		\word{:} so5  \word{DUP} \word{IF} \word{SWAP} \word{EXECUTE} \word{THEN} \word{;}

		\test{\word{Sq} w1" wid1 \word{@} \word{SEARCH-WORDLIST} so5}{-1  1234} \\
		\test{\word{Sq} w1" wid2 \word{@} \word{SEARCH-WORDLIST} so5}{ 1 -9876}

		\word{:} c"w1" \word{Cq} w1" \word{;} \\
		\test{alsowid2 c"w1" \word{FIND} so5}{ 1 -9876} \\
		\test{\word{PREVIOUS} c"w1" \word{FIND} so5}{-1  1234}
	\end{testing}
\end{worddef}


\begin{worddef}{1550}{FIND}
\item Extend the semantics of \wref{core:FIND}{FIND} to be:

	\stack{c-addr}{c-addr 0 | xt 1 | xt -1}

	Find the definition named in the counted string at \param{c-addr}.
	If the definition is not found after searching all the word lists
	in the search order, return \param{c-addr} and zero. If
	the definition is found, return \param{xt}. If the definition is
	immediate, also return one (\param{1}); otherwise also return
	minus-one (\param{-1}). For a given string, the values returned
	by \word{FIND} while compiling may differ from those returned
	while not compiling.

\see \xref[3.4.2 Finding definition names]{usage:find},
	\wref{core:'}{'},
	\wref{core:FIND}{FIND},
	\wref{core:POSTPONE}{POSTPONE},
	\wref{core:[']}{[']}.

	\begin{implement} % I.16.6.1.1550 FIND
		\textdf{Assuming \texttt{\#order} and \texttt{context} are defined as per \iref{search:GET-ORDER}{}.}

		\begin{tabbing}
		\tab \= \tab \= \tab \= \tab \= 2swap 2drop leave \tab \= \kill
		\word{:} \word{FIND} \word{p} c-addr -{}- c-addr 0 | xt 1 | xt -1 ) \+ \\
			0 														\>\>\>\>	\word{p} c-addr 0 ) \\
			\#order \word{@} 0 \word{qDO} \+ \\
				\word{OVER} \word{COUNT}							\>\>\>	\word{p} c-addr 0 c-addr' u ) \\
				\word{I} \word{CELLS} context \word{+} \word{@}		\>\>\>	\word{p} c-addr 0 c-addr' u wid ) \\
				\word{SEARCH-WORDLIST}								\>\>\>	\word{p} c-addr 0; 0 | w 1 | q -1 ) \\
				\word{qDUP} \word{IF}								\>\>\>	\word{p} c-addr 0; w 1 | w -1 ) \+ \\
				\>	\word{2SWAP} \word{2DROP} \word{LEAVE}			\>		\word{p} w 1 | w -1 )  \\
				\word{THEN}											\>\>	\word{p} c-addr 0 ) \- \\
			\word{LOOP}												\>\>\>	\word{p} c-addr 0 | w 1 | w -1 ) \- \\
		\word{;}
		\end{tabbing}
	\end{implement}

	\begin{testing} % T.16.6.1.1550 FIND
		\ttfamily
		\word{:} c"dup" \word{Cq} DUP" \word{;} \\
		\word{:} c".("~ \word{Cq} .("~ \word{;} \\
		\word{:} c"x"~~ \word{Cq} unknown word" \word{;}

		\test{c"dup" \word{FIND}}{xt  \word{@} -1} \\
		\test{c".("  \word{FIND}}{xti \word{@}  1} \\
		\test{c"x"   \word{FIND}}{c"x"   0}
	\end{testing}
\end{worddef}


\begin{worddef}{1595}{FORTH-WORDLIST}
\item \stack{}{wid}

	Return \param{wid}, the identifier of the word list that includes
	all standard words provided by the implementation. This word list
	is initially the compilation word list and is part of the initial
	search order.

	\begin{testing} % T.16.6.2.1595 FORTH-WORDLIST
		\test{\word{FORTH-WORDLIST} wid1 \word{!}}{}
	\end{testing}
\end{worddef}


\begin{worddef}{1643}{GET-CURRENT}
\item \stack{}{wid}

	Return \param{wid}, the identifier of the compilation word list.
\end{worddef}


\begin{worddef}{1647}{GET-ORDER}
\item \stack{}{wid_n {\ldots} wid_1 n}

	Returns the number of word lists \param{n} in the search order
	and the word list identifiers \param{wid_n} {\ldots} \param{wid_1}
	identifying these word lists. \param{wid_1} identifies the word
	list that is searched first, and \param{wid_n} the word list that
	is searched last. The search order is unaffected.

	\begin{implement} % I.16.6.1.1647 GET-ORDER
		\textdf{Here is a very simple search order implementation:}

		\word{VARIABLE} \#order

		\word{CREATE} context  ~ 16 \word{p} wordlists ) \word{CELLS} \word{ALLOT}

		\word{:} \word{GET-ORDER} \word{p} -{}- wid1 {\ldots} widn n ) \\
		\tab \#order \word{@} 0 \word{qDO} \\
		\tab[2] \#order \word{@} \word{I} \word{-} \word{1-} \word{CELLS} context \word{+} \word{@} \\
		\tab \word{LOOP} \\
		\tab \#order \word{@} \\
		\word{;}
	\end{implement}
\end{worddef}


\begin{worddef}{2192}{SEARCH-WORDLIST}
\item \stack{c-addr u wid}{0 | xt 1 | xt -1}

	Find the definition identified by the string \param{c-addr u} in
	the word list identified by \param{wid}. If the definition is not
	found, return zero. If the definition is found, return its
	execution token \param{xt} and one (\param{1}) if the definition is
	immediate, minus-one (\param{-1}) otherwise.

\see \rref{search:SEARCH-WORDLIST}{}.

	\begin{rationale} % A.16.6.1.2192 SEARCH-WORDLIST
		When \word{SEARCH-WORDLIST} fails to find the word, it does
		not return the string, unlike \word{FIND}. This is in
		accordance with the general principle that Forth words consume
		their arguments.
	\end{rationale}

	\begin{testing} % T.16.6.1.2192 SEARCH-WORDLIST
		\ttfamily
		\word{ONLY} \word{FORTH} \word{DEFINITIONS} \\
		\word{VARIABLE} xt~ \word{'} \word{DUP} xt~ \word{!} \\
		\word{VARIABLE} xti \word{'} \word{.p}~ xti \word{!} \word{bs} \textdf{Immediate word}

		\test{\word{Sq} DUP" wid1 \word{@} \word{SEARCH-WORDLIST}}{xt  \word{@} -1} \\
		\test{\word{Sq} .("  wid1 \word{@} \word{SEARCH-WORDLIST}}{xti \word{@}  1} \\
		\test{\word{Sq} DUP" wid2 \word{@} \word{SEARCH-WORDLIST}}{       0}
	\end{testing}
\end{worddef}


\begin{worddef}{2195}{SET-CURRENT}
\item \stack{wid}{}

	Set the compilation word list to the word list identified by
	\param{wid}.

	\begin{testing} % T.16.6.1.2195 SET-CURRENT
		\test{\word{GET-CURRENT}}{wid1 \word{@}}

		\test{\word{WORDLIST} wid2 \word{!}}{} \\
		\test{wid2 \word{@} \word{SET-CURRENT}}{} \\
		\test{\word{GET-CURRENT}}{wid2 \word{@}}

		\test{wid1 \word{@} \word{SET-CURRENT}}{}
	\end{testing}
\end{worddef}


\begin{worddef}{2197}{SET-ORDER}
\item \stack{wid_n {\ldots} wid_1 n}{}

	Set the search order to the word lists identified by
	\param{wid_n} {\ldots} \param{wid_1}. Subsequently, word list
	\param{wid_1} will be searched first, and word list \param{wid_n}
	searched last. If \param{n} is zero, empty the search order. If
	\param{n} is minus one, set the search order to the
	implementation-defined minimum search order. The minimum search
	order shall include the words \word{FORTH-WORDLIST} and
	\word{SET-ORDER}. A system shall allow \param{n} to be at least
	eight.

	\begin{implement} % I.16.6.1.2197 SET-ORDER
		\textdf{This is the complement of \iref{search:GET-ORDER}{}.}

		\word{:} \word{SET-ORDER} \word{p} wid1 {\ldots} widn n -{}0 ) \\
		\tab \word{DUP} -1 \word{=} \word{IF} \\
		\tab[2] \word{DROP} \arg{\textdf{push system default word lists and n}} \\
		\tab \word{THEN} \\
		\tab \word{DUP} \#order \word{!} \\
		\tab 0 \word{qDO} ~ \word{I} \word{CELLS} context \word{+} \word{!} ~ \word{LOOP} \\
		\word{;}
	\end{implement}

	\begin{testing} % T.16.6.1.2197 SET-ORDER
		\ttfamily
		\test{\word{GET-ORDER} \word{OVER}     }{\word{GET-ORDER} wid1 \word{@}} \\% \word{bs} \textdf{Forth wordlist at top} \\
		\test{\word{GET-ORDER} \word{SET-ORDER}}{} \\% \tab \word{bs} \textdf{Effectively noop} \\
		\test{\word{GET-ORDER}          }{get-orderlist} % \tab \word{bs} \textdf{Check nothing changed} \\

		\test{get-orderlist \word{DROP} get-orderList \word{2*} \word{SET-ORDER}}{} \\
		\test{\word{GET-ORDER}}{get-orderlist \word{DROP} get-orderList \word{2*}} \\
		\test{get-orderlist \word{SET-ORDER} \word{GET-ORDER}}{get-orderlist}

		\word{:} so2a \word{GET-ORDER} get-orderlist \word{SET-ORDER} \word{;} \\
		\word{:} so2 0 \word{SET-ORDER} so2a \word{;}

		\test{so2}{0}	\tab[1.5] \word{bs} \textdf{0 SET-ORDER leaves an empty search order}

		\word{:} so3 -1 \word{SET-ORDER} so2a \word{;} \\
		\word{:} so4 \word{ONLY} so2a \word{;}

		\test{so3}{so4}	\tab \word{bs} \textdf{-1 SET-ORDER is the same as ONLY}
	\end{testing}
\end{worddef}

\enlargethispage{2ex}
\begin{worddef}{2460}{WORDLIST}
\item \stack{}{wid}

	Create a new empty word list, returning its word list identifier
	\param{wid}. The new word list may be returned from a pool of
	preallocated word lists or may be dynamically allocated in data
	space. A system shall allow the creation of at least 8 new word
	lists in addition to any provided as part of the system.
\end{worddef}


\subsection{Search-Order extension words} % 16.6.2
\extended

\begin{worddef}{0715}{ALSO}
\item \stack{}{}

	Transform the search order consisting of \param{wid_n}, {\ldots}
	\param{wid_2}, \param{wid_1} (where \param{wid_1} is searched
	first) into \param{wid_n}, {\ldots} \param{wid_2}, \param{wid_1},
	\param{wid_1}. An ambiguous condition exists if there are too
	many word lists in the search order.

	\begin{implement} % I.16.6.2.0715 ALSO
		\word{:} \word{ALSO} \word{p} -{}- ) \\
		\tab \word{GET-ORDER} \word{OVER} \word{SWAP} \word{1+} \word{SET-ORDER} \\
		\word{;}
	\end{implement}

	\begin{testing} % T.16.6.2.0715 ALSO
		\test{\word{ALSO} \word{GET-ORDER} \word{ONLY}}{get-orderlist \word{OVER} \word{SWAP} \word{1+}}
	\end{testing}
\end{worddef}

\pagebreak
\begin{worddef}{1590}{FORTH}
\item \stack{}{}

	Transform the search order consisting of \param{wid_n}, {\ldots}
	\param{wid_2}, \param{wid_1} (where \param{wid_1} is searched
	first) into \param{wid_n}, {\ldots} \param{wid_2},
	\param{wid_{\word{FORTH-WORDLIST}}}.

	\begin{implement} % I.16.6.2.1590 FORTH
	\word{:} (wordlist) \word{p} wid "<name>" -{}- ; ) \\
	\tab \word{CREATE} \word{,} \\
	\tab \word{DOES} \\
	\tab[2] \word{@}  \word{toR} \\
	\tab[2] \word{GET-ORDER} \word{NIP} \\
	\tab[2] \word{Rfrom} \word{SWAP} \word{SET-ORDER} \\
	\word{;}

	\word{FORTH-WORDLIST} (wordlist) \word{FORTH}
	\end{implement}
\end{worddef}


\begin{worddef}{1965}{ONLY}
\item \stack{}{}

	Set the search order to the implementation-defined minimum search
	order. The minimum search order shall include the words
	\word{FORTH-WORDLIST} and \word{SET-ORDER}.

	\begin{implement} % A.16.6.2.1965 ONLY
		\word{:} \word{ONLY} \word{p} -{}- ) -1 \word{SET-ORDER} \word{;}
	\end{implement}

	\begin{testing} % T.16.6.2.1965 ONLY
		\ttfamily
		\test{\word{ONLY} \word{FORTH} \word{GET-ORDER}}{get-orderlist}

		\word{:} so1 \word{SET-ORDER} {;} \word{bs} \textdf{In case it is unavailable in the forth wordlist}

		\test{\word{ONLY} \word{FORTH-WORDLIST} 1 \word{SET-ORDER} get-orderlist so1}{} \\
		\test{\word{GET-ORDER}}{get-orderlist}
	\end{testing}
\end{worddef}


\begin{worddef}{1985}{ORDER}
\item \stack{}{}

	Display the word lists in the search order in their search order
	sequence, from first searched to last searched. Also display the
	word list into which new definitions will be placed. The display
	format is implementation dependent.

	\word{ORDER} may be implemented using pictured numeric output
	words. Consequently, its use may corrupt the transient region
	identified by \word[core]{num-end}.

\see \xref[3.3.3.6 Other Transient Regions]{usage:transient}.

	\begin{testing} % T.16.6.2.1985 ORDER
		\ttfamily
		\word{CR} \word{.p} ONLY FORTH DEFINITIONS search order and compilation list) \word{CR} \\
		\test{\word{ONLY} \word{FORTH} \word{DEFINITIONS} \word{ORDER}}{}

		\word{CR} \word{.p} Plus another unnamed wordlist at head of search order) \word{CR} \\
		\test{alsowid2 \word{DEFINITIONS} \word{ORDER}}{}
	\end{testing}
\end{worddef}


\begin{worddef}{2037}{PREVIOUS}
\item \stack{}{}

	Transform the search order consisting of \param{wid_n}, {\ldots}
	\param{wid_2}, \param{wid_1} (where \param{wid_1} is searched
	first) into \param{wid_n}, {\ldots} \param{wid_2}. An ambiguous
	condition exists if the search order was empty before
	\word{PREVIOUS} was executed.

	\begin{implement} % I.16.6.2.2037 PREVIOUS
		\word{:} \word{PREVIOUS} \word{p} -{}- )
			 \word{GET-ORDER} \word{NIP} \word{1-} \word{SET-ORDER}
		\word{;}
	\end{implement}
\end{worddef}
