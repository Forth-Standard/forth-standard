\chapter{The optional Locals word set} % 13
\wordlist{local}

\begin{intro}
\section{The optional Locals word set} % A.13
\label{rat:locals}

The Technical Committee has had a problem with locals. It has been
argued forcefully that ANS Forth should say nothing about locals
since:

\begin{itemize}
\item there is no clear accepted practice in this area;
\item not all Forth programmers use them or even know what they are;
and
\item few implementations use the same syntax, let alone the same
	broad usage rules and general approaches.
\end{itemize}

It has also been argued, it would seem equally forcefully, that the
lack of any standard approach to locals is precisely the reason for
this lack of accepted practice since locals are at best non-trivial
to implement in a portable and useful way. It has been further argued
that users who have elected to become dependent on locals tend to be
locked into a single vendor and have little motivation to join the
group that it is hoped will ``broadly accept'' ANS Forth unless the
Standard addresses their problems.

Since the Technical Committee has been unable to reach a strong
consensus on either leaving locals out or on adopting any particular
vendor's syntax, it has sought some way to deal with an issue that it
has been unable to simply dismiss. Realizing that no single mechanism
or syntax can simultaneously meet the desires expressed in all the
locals proposals that have been received, it has simplified the
problem statement to be to define a locals mechanism that:

\begin{itemize}
\item is independent of any particular syntax;
\item is user extensible;
\item enables use of arbitrary identifiers,
	local in scope to a single definition;
\item supports the fundamental cell size data types of Forth;
and
\item works consistently, especially with respect to
	re-entrancy and recursion.
\end{itemize}

This appears to the Technical Committee to be what most of those who
actively use locals are trying to achieve with them, and it is at
present the consensus of the Technical Committee that if ANS Forth has
anything to say on the subject this is an acceptable thing for it to
say.

This approach, defining \word{LOCAL}, is proposed as one that can be
used with a small amount of user coding to implement some, but not all,
of the locals schemes in use. The following coding examples illustrate
how it can be used to implement two syntaxes.

\setwordlist{core}
\begin{itemize}
\item The syntax defined by this Standard and used in the systems of
	Creative Solutions, Inc.:

	\begin{quote}\ttfamily
	\word{:} \word[local]{LOCALS} \word{p} "name{\ldots}name |" -- ) \\
	\tab \word{BEGIN} \\
	\tab~ \word{BL} \word{WORD} ~ \word{COUNT} \word{OVER} \word{C@} \\
	\tab~ \word{[CHAR]} | \word{-} \word{OVER} 1 \word{-} \word{OR} ~ \word{WHILE} \\
	\tab~ \word[local]{LOCAL} \\
	\tab \word{REPEAT} \word{2DROP} ~ 0 0 \word[local]{LOCAL} \\
	\word{;} \word{IMMEDIATE}

	\word{:} EXAMPLE \word{p} n -- n**2 n**3 ) \\
	\tab \word[local]{LOCALS} N |
		~ N \word{DUP} N \word{*} ~\word{DUP} N \word{*}
	\word{;}
	\end{quote}

\item A proposed syntax: ( \texttt{LOCAL} \emph{name} ) with
	additional usage rules:

	\begin{quote}\ttfamily
	\word{:} LOCAL \word{p} "name" -- )
		\word{BL} \word{WORD} \word{COUNT} \word[local]{LOCAL}
	\word{;} ~\word{IMMEDIATE}

	\word{:} END-LOCALS \word{p} -- ) ~ 0 0 \word[local]{LOCAL}
	\word{;} ~\word{IMMEDIATE}

	\word{:} EXAMPLE \word{p} n -- n n**2 n**3 ) \\
	\tab LOCAL N ~END-LOCALS ~
		N \word{DUP} N \word{*} ~ \word{DUP} N \word{*}
	\word{;}
	\end{quote}
\end{itemize}

Other syntaxes can be implemented, although some will admittedly
require considerably greater effort or in some cases program
conversion. Yet other approaches to locals are completely incompatible
due to gross differences in usage rules and in some cases even scope
identifiers. For example, the complete local scheme in use at Johns
Hopkins had elaborate semantics that cannot be duplicated in terms of
this model.

To reinforce the intent of section \ref{wordlist:local}, here are
two examples of actual use of locals. The first illustrates correct
usage:

\begin{enumerate}
\item[a)] \begin{quote}\ttfamily
	\word{:} \{	\word{p} "name {\ldots} \}" -- ) \\
	\tab \word{BEGIN} ~\word{BL} \word{WORD} \word{COUNT} \\
	\tab~~ \word{OVER} \word{C@} \word{[CHAR]} \} \word{-} ~
		\word{OVER} 1 \word{-} ~
		\word{OR} \word{WHILE} \\
	\tab~~ \word[local]{LOCAL} \\
	\tab \word{REPEAT} \word{2DROP} 0 0 \word[local]{LOCAL}\\
	\word{;} \word{IMMEDIATE}
	\end{quote}

\item[b)] \begin{quote}\ttfamily
	\word{:} JOE \word{p} a b c -- n ) \\
	\tab \word{toR} \word{2*} \word{Rfrom} \word{2DUP} \word{+} 0 \\
	\tab \{ ANS 2B+C C 2B A \} \\
	\tab 2 0 \word{DO}
		~1 ANS \word{+} \word{I} \word{+} \word[local]{TO} ANS
		~ANS \word{d} \word{CR}
	~\word{LOOP} \\
	\tab ANS \word{d} 2B+C \word{d} C \word{d} 2B \word{d} A \word{d} \word{CR} ANS \\
	\word{;}
	\end{quote}

\item[c)] \begin{quote}\ttfamily
	100 300 10 JOE \word{d}
	\end{quote}
\end{enumerate}

The word \texttt{\{} at a) defines a local declaration syntax that
surrounds the list of locals with braces. It doesn't do anything fancy,
such as reordering locals or providing initial values for some of them,
so locals are initialized from the stack in the default order. The
definition of \texttt{JOE} at b) illustrates a use of this syntax. Note
that work is performed at execution time in that definition before
locals are declared. It's OK to use the return stack as long as whatever
is placed there is removed before the declarations begin.

Note that before declaring locals, \texttt{B} is doubled, a
subexpression (\texttt{2B+C}) is computed, and an initial value (zero)
for \texttt{ANS} is provided. After locals have been declared,
\texttt{JOE} proceeds to use them. Note that locals may be accessed and
updated within do-loops. The effect of interpreting line c) is to display
the following values:

\begin{quote}
	\texttt{1} (\texttt{ANS} the first time through the loop),\\
	\texttt{3} (\texttt{ANS} the second time), \\
	\texttt{3} (\texttt{ANS}),
		\texttt{610} (\texttt{2B+C}),
		\texttt{10}  (\texttt{C}),
		\texttt{600} (\texttt{2B}),
		\texttt{100} (\texttt{A}), and \\
	\texttt{3} (\texttt{ANS} left on the stack by \texttt{JOE}).
\end{quote}

The \emph{names} of the locals vanish after \texttt{JOE} has been
compiled. The \emph{storage and meaning} of locals appear when
\texttt{JOE}'s locals are declared and vanish as \texttt{JOE} returns
to its caller at \word{;} (semicolon).

A second set of examples illustrates various things that break the
rules. We assume that the definitions of \texttt{LOCAL} and
\texttt{END-LOCALS} above are present, along with \texttt{\{} from
the preceding example.

\begin{enumerate}
\item[d)] \begin{quote}\ttfamily
	\word{:} ZERO ~ 0 \word{POSTPONE} \word{LITERAL} \word{POSTPONE} LOCAL
	\word{;} \word{IMMEDIATE}
	\end{quote}

\item[e)] \begin{quote}\ttfamily
	\word{:} MOE ~\word{p} a b ) \\
	\tab ZERO TEMP ~LOCAL B ~\word{1+} LOCAL A+ ~ZERO ANSWER \word{;}
	\end{quote}

\item[f)] \begin{quote}\ttfamily
	\word{:} BOB~ \word{p} a b c d ) ~\{ D C \} ~\{ B A \} \word{;}
	\end{quote}
\end{enumerate}

Here are two definitions with various violations of rule
\ref{locals:rules}a. In e) the declaration of \texttt{TEMP} is legal
and creates a local whose initial value is zero. It's OK because the
executable code that \texttt{ZERO} generates precedes the first use of
\word[local]{LOCAL} in the definition. However, the \word{1+}
preceding the declaration of \texttt{A+} is illegal. Likewise the use
of \texttt{ZERO} to define \texttt{ANSWER} is illegal because it
generates executable code between uses of \word[local]{LOCAL}.
Finally, \texttt{MOE} terminates illegally (no \texttt{END-LOCALS}).
\texttt{BOB} inf) violates the rule against declaring two sets of
locals.
% \replace[1]{inf) violates}{inviolate}

\begin{enumerate}
\item[g)] \begin{quote}\ttfamily
	\word{:} ANN \word{p} a b -- b )
		\word{DUP} \word{toR} \word{DUP} \word{IF}
		\{ B A \} \word{THEN} \word{Rfrom} \word{;}
	\end{quote}

\item[h)] \begin{quote}\ttfamily
	\word{:} JANE \word{p} a b -- n )
		\{ B A \} A B \word{+} \word{toR} A B \word{-} ~\word{Rfrom} \word{/} \word{;}
	\end{quote}
\end{enumerate}

\texttt{ANN} in g) violates two rules. The \word{IF} {\ldots} \word{THEN}
around the declaration of its locals violates \ref{locals:rules}b, and
the copy of \texttt{B} left on the return stack before declaring locals
violates \ref{locals:rules}c. \texttt{JANE} in h) violates
\ref{locals:rules}d by accessing locals after placing the sum of
\texttt{A} and \texttt{B} on the return stack without first removing
that sum.

\begin{enumerate}
\item[i)] \begin{quote}\ttfamily
	\word{:} CHRIS \word{p} a b) \\
	\tab \{ B A \} \word{[']} A \word{EXECUTE}
		~5 \word{[']} B \word{toBODY} \word{!}
		~\word{[} \word{'} A \word{]} \word{LITERAL} LEE
	\word{;}
	\end{quote}
\end{enumerate}

\texttt{CHRIS} in i) illustrates three violations of
\ref{locals:rules}e. The attempt to \word{EXECUTE} the local called
\texttt{A} is inconsistent with some implementations. The store into
\texttt{B} via \word{toBODY} is likely to cause tragic results with
many implementations; moreover, if locals are in registers they can't
be addressed as memory no matter what is written.

The third violation, in which an execution token for a definition's
local is passed as an argument to the word \texttt{LEE}, would, if
allowed, have the unpleasant implication that \texttt{LEE} could
\word{EXECUTE} the token and obtain a value for \texttt{A} from the
particular execution of \texttt{CHRIS} that called \texttt{LEE} this
time.

\setwordlist{local}
\end{intro}

\section{Introduction} % 13.1

See: \xref[Annex A.13 The Locals Word Set]{rat:locals}.

\section{Additional terms and notation} % 13.2

None.

\section{Additional usage requirements} % 13.3

\begin{intro}
\subsection{Additional usage requirements} % A.13.3

Rule \ref{locals:rules}d could be relaxed without affecting the
integrity of the rest of this structure. \ref{locals:rules}c could
not be.

\ref{locals:rules}b forbids the use of the data stack for local
storage because no usage rules have been articulated for programmer
users in such a case. Of course, if the data stack is somehow employed
in such a way that there are no usage rules, then the locals are
invisible to the programmer, are logically not on the stack, and the
implementation conforms.

The minimum required number of locals can (and should) be adjusted to
minimize the cost of compliance for existing users of locals.

Access to previously declared local variables is prohibited by Section
\ref{locals:rules}d until any data placed onto the return stack by the
application has been removed, due to the possible use of the return
stack for storage of locals.

Authorization for a Standard Program to manipulate the return stack
(e.g., via \word[core]{toR} \word[core]{Rfrom}) while local variables are
active overly constrains implementation possibilities. The consensus
of users of locals was that Local facilities represent an effective
functional replacement for return stack manipulation, and restriction
of standard usage to only one method was reasonable.

Access to Locals within \word[core]{DO}{\ldots}\word[core]{LOOP}s is
expressly permitted as an additional requirement of conforming systems
by Section \ref{locals:rules}g. Although words, such as \word{LOCAL},
written by a System Implementor, may require inside knowledge of the
internal structure of the return stack, such knowledge is not required
of a user of compliant Forth systems.
\end{intro}


\subsection{Locals} % 13.3.1

A local is a data object whose execution semantics shall return its
value, whose scope shall be limited to the definition in which it is
declared, and whose use in a definition shall not preclude reentrancy
or recursion.

\subsection{Environmental queries} % 13.3.2

Append table \ref{local:env} to table \ref{table:env}.

See: \xref[3.2.6 Environmental queries]{usage:env}.

\begin{table}[h]
  \begin{center}
	\caption{Environmental Query Strings}
	\label{local:env}
	\begin{tabular}{p{9em}rcp{0.42\textwidth}}
		\hline\hline
		\multicolumn{2}{l}{String \hfill Value data type} & Constant? & Meaning \\
		\hline
		\texttt{\#LOCALS}		& \emph{n}	& yes	&
			maximum number of local variables in a definition \\
		\texttt{LOCALS}	& \emph{flag}	& no	&
			locals word set present \\
		\texttt{LOCALS-EXT}	& \emph{flag}	& no	&
			locals extensions word set present \\
		\hline\hline
	\end{tabular}
  \end{center}
\end{table}

\subsection{Processing locals} % 13.3.3
\label{local:locals}

To support the locals word set, a system shall provide a mechanism
to receive the messages defined by \word{LOCAL} and respond as
described here.

During the compilation of a definition after \word[core]{:} (colon),
\word[core]{:NONAME}, or \word[core]{DOES}, a program may begin
sending local identifier messages to the system. The process shall
begin when the first message is sent. The process shall end when the
``last local'' message is sent. The system shall keep track of the
names, order, and number of identifiers contained in the complete
sequence.

\subsubsection{Compilation semantics} % 13.3.3.1

The system, upon receipt of a sequence of local-identifier messages,
shall take the following actions at compile time:
\begin{enumerate}
\item Create temporary dictionary entries for each of the
	identifiers passed to \word{LOCAL}, such that each identifier
	will behave as a \emph{local}. These temporary dictionary
	entries shall vanish at the end of the definition, denoted by
	\word[core]{;} (semicolon), \word[tools]{;CODE}, or
	\word[core]{DOES}. The system need not maintain these
	identifiers in the same way it does other dictionary entries
	as long as they can be found by normal dictionary searching
	processes. Furthermore, if the Search-Order word set is present,
	local identifiers shall always be searched before any of the
	word lists in any definable search order, and none of the
	Search-Order words shall change the locals' privileged position
	in the search order. Local identifiers may reside in mass storage.

\item For each identifier passed to \word{LOCAL}, the system shall
	generate an appropriate code sequence that does the following at
	execution time:
	\begin{enumerate}
	\item Allocate a storage resource adequate to contain the value
		of a local. The storage shall be allocated in a way that
		does not preclude re-entrancy or recursion in the definition
		using the local.
	\item Initialize the value using the top item on the data stack.
		If more than one local is declared, the top item on the
		stack shall be moved into the first local identified, the
		next item shall be moved into the second, and so on.
	\end{enumerate}

	The storage resource may be the return stack or may be
	implemented in other ways, such as in registers. The storage
	resource shall not be the data stack. Use of locals shall not
	restrict use of the data stack before or after the point of
	declaration.

\item Arrange that any of the legitimate methods of terminating
	execution of a definition, specifically \word[core]{;}
	(semicolon), \word[tools]{;CODE}, \word[core]{DOES} or
	\word[core]{EXIT}, will release the storage resource allocated
	for the locals, if any, declared in that definition.
	\word[core]{ABORT} shall release all local storage resources,
	and \word[exception]{CATCH} / \word[exception]{THROW} (if
	implemented) shall release such resources for all definitions
	whose execution is being terminated.

\item Separate sets of locals may be declared in defining words
	before \word[core]{DOES} for use by the defining word, and
	after \word[core]{DOES} for use by the word defined.
\end{enumerate}

A system implementing the Locals word set shall support the
declaration of at least eight locals in a definition.


\subsubsection{Syntax restrictions} % 13.3.3.2
\label{locals:rules}

Immediate words in a program may use \word{LOCAL} to implement
syntaxes for local declarations with the following restrictions:

\begin{enumerate}
\item A program shall not compile any executable code into the
	current definition between the time \word{LOCAL} is executed
	to identify the first local for that definition and the time of
	sending the single required ``last local'' message;

\item The position in program source at which the sequence of
	\word{LOCAL} messages is sent, referred to here as the point
	at which locals are declared, shall not lie within the scope of
	any control structure;

\item Locals shall not be declared until values previously placed on
	the return stack within the definition have been removed;

\item After a definition's locals have been declared, a program may
	place data on the return stack. However, if this is done,
	locals shall not be accessed until those values have been
	removed from the return stack;

\item Words that return execution tokens, such as \word[core]{'}
	(tick), \word[core]{[']}, or \word[core]{FIND}, shall not be
	used with local names;

\item A program that declares more than eight locals in a single
	definition has an environmental dependency;

\item Locals may be accessed or updated within control structures,
	including do-loops;

\item Local names shall not be referenced by \word[core]{POSTPONE}
	and \word[core]{[COMPILE]}.
\end{enumerate}

See: \xref[3.4 The Forth text interpreter]{usage:command}.


\section{Additional documentation requirements} % 13.4

\subsection{System documentation} % 13.4.1

\subsubsection{Implementation-defined options} % 13.4.1.1
\begin{itemize}
\item maximum number of locals in a definition
	(\xref[13.3.3 Processing locals]{local:locals},
	 \wref{local:LOCALS}{LOCALS|}).
\end{itemize}

\subsubsection{Ambiguous conditions} % 13.4.1.2
\begin{itemize}
\item executing a named \emph{local} while in interpretation state
	(\wref{local:LOCAL}{(LOCAL)});
\item \remove{x:to}{\emph{name} not defined by \word[core]{VALUE} or \word{LOCAL}
	(\wref{local:TO}{TO}).}
\end{itemize}

\subsubsection{Other system documentation} % 13.4.1.3
\begin{itemize}
\item no additional requirements.
\end{itemize}

\subsection{Program documentation} % 13.4.2

\subsubsection{Environmental dependencies} % 13.4.2.1
\begin{itemize}
\item declaring more than eight locals in a single definition
	(\xref[13.3.3 Processing locals]{local:locals}).
\end{itemize}

\subsubsection{Other program documentation} % 13.4.2.2
\begin{itemize}
\item no additional requirements.
\end{itemize}


\section{Compliance and labeling} % 13.5

\subsection{ANS Forth systems} % 13.5.1

The phrase ``Providing the Locals word set'' shall be appended to
the label of any Standard System that provides all of the Locals
word set.

The phrase ``Providing \emph{name(s)} from the Locals Extensions
word set'' shall be appended to the label of any Standard System
that provides portions of the Locals Extensions word set.

The phrase ``Providing the Locals Extensions word set'' shall be
appended to the label of any Standard System that provides all of
the Locals and Locals Extensions word sets.

\subsection{ANS Forth programs} % 13.5.2

The phrase ``Requiring the Locals word set'' shall be appended to
the label of Standard Programs that require the system to provide
the Locals word set.

The phrase ``Requiring \emph{name(s)} from the Locals Extensions
word set'' shall be appended to the label of Standard Programs that
require the system to provide portions of the Locals Extensions word
set.

The phrase ``Requiring the Locals Extensions word set'' shall be
appended to the label of Standard Programs that require the system
to provide all of the Locals and Locals Extensions word sets.


\section{Glossary} % 13.6

\begin{intro}
\subsection{Glossary} % A.13.6
\end{intro}

\subsection{Locals words} % 13.6.1

\begin{worddef}[LOCAL]{0086}{(LOCAL)}[paren-local-paren]
\interpret
	Interpretation semantics for this word are undefined.

\execute
	\stack{c-addr u}{}

	When executed during compilation, \word{LOCAL} passes a
	message to the system that has one of two meanings. If \param{u}
	is non-zero, the message identifies a new \emph{local} whose
	definition name is given by the string of characters identified
	by \param{c-addr u}. If \param{u} is zero, the message is ``last
	local'' and \param{c-addr} has no significance.

	The result of executing \word{LOCAL} during compilation of a
	definition is to create a set of named local identifiers, each
	of which is a definition name, that only have execution
	semantics within the scope of that definition's source.

\execute[local]
	\stack{}{x}

	Push the local's value, \param{x}, onto the stack. The local's
	value is initialized as described in \xref[13.3.3 Processing
	locals]{local:locals} and may be changed by preceding the local's
	name with \word{TO}. An ambiguous condition exists when local is
	executed while in interpretation state.

\note
	This word does not have special compilation semantics in the
	usual sense because it provides access to a system capability
	for use by other user-defined words that do have them. However,
	the locals facility as a whole and the sequence of messages
	passed defines specific usage rules with semantic implications
	that are described in detail in section
	\xref[13.3.3 Processing locals]{local:locals}.

\note
	This word is not intended for direct use in a definition to
	declare that definition's locals. It is instead used by system
	or user compiling words. These compiling words in turn define
	their own syntax, and may be used directly in definitions to
	declare locals. In this context, the syntax for \word{LOCAL}
	is defined in terms of a sequence of compile-time messages and
	is described in detail in section \xref[13.3.3 Processing
	locals]{local:locals}.

\note
	\remove{x:to}{%
	The Locals word set modifies the syntax and semantics of
	\wref{core:TO}{TO} as defined in the Core Extensions word set.}

\runtime[\word{TO} \param{local}]
	\stack{\place{x:to}{x}}{}

	\place{x:to}{Associate the value \param{x} with the local value
	\param{local}.}

\see \xref[3.4 The Forth text interpreter]{usage:command}
	\place{x:to}{ and \wref{core:TO}{TO}.}
\end{worddef}


\begin{worddef}{2295}{TO}
\item \remove{x:to}{Extend the semantics of \wref{core:TO}{TO} to be:}

\interpret
	\stack{\remove{x:to}{x "<spaces>name"}}{}

	\remove{x:to}{
	Skip leading spaces and parse \param{name} delimited by a space.
	Store \param{x} in \param{name}. An ambiguous condition exists if
	\param{name} was not defined by \word[core]{VALUE}.}

\compile
	\stack{\remove{x:to}{"<spaces>name"}}{}

	\remove{x:to}{
	Skip leading spaces and parse \param{name} delimited by a space.
	Append the run-time semantics given below to the current
	definition. An ambiguous condition exists if \param{name} was
	not defined by either \word[core]{VALUE} or \word{LOCAL}.}

\runtime
	\stack{\remove{x:to}{x}}{}

	\remove{x:to}{Store \param{x} in \param{name}.}

\note
	\remove{x:to}{
	An ambiguous condition exists if either \word[core]{POSTPONE} or
	\word[core]{[COMPILE]} is applied to \word{TO}.}

\see
	\xref[3.4.1 Parsing]{usage:parsing},
	\remove{x:to}{
	\wref{core:TO}{TO},
	\wref{core:VALUE}{VALUE},
	\wref{local:LOCAL}{(LOCAL)}.}

	\begin{defer}
	\rationale % A.13.6.1.2295 TO
		\remove{x:to}{Typical use: \texttt{x} \word[local]{TO} \texttt{name}}

		\remove{x:to}{See: \rref{core:TO}{TO}.}
	\end{defer}
\end{worddef}


\subsection{Locals extension words} % 13.6.2
\extended

\begin{worddef}[LOCALS]{1795}{LOCALS|}[locals-bar]
\interpret
	Interpretation semantics for this word are undefined.

\compile
	\stack{"<spaces>name_1" "<spaces>name_2" {\ldots} "<spaces>name_n" "|"}{}

	Create up to eight local identifiers by repeatedly skipping
	leading spaces, parsing \param{name}, and executing
	\wref{local:LOCAL}{(LOCAL)}. The list of locals to be defined
	is terminated by \param{|}. Append the run-time semantics given
	below to the current definition.

\runtime
	\stack{x_n {\ldots} x_2 x_1}{}

	Initialize up to eight local identifiers as described in
	\wref{local:LOCAL}{(LOCAL)}, each of which takes as its
	initial value the top stack item, removing it from the stack.
	Identifier \param{name_1} is initialized with \param{x_1},
	identifier \param{name_2} with \param{x_2}, etc. When invoked,
	each local will return its value. The value of a local may be
	changed using \replace{x:to}{\wref{local:TO}{TO}}{\wref{core:TO}{TO}}.

	\begin{defer}
	\rationale % A.13.6.2.1795 LOCALS|
		A possible implementation of this word and an example of usage
		is given in \ref{rat:locals}, above. It is intended as an
		example only; any implementation yielding the described
		semantics is acceptable.
	\end{defer}
\end{worddef}
