% !TeX root = forth.tex

\chapter{The optional Block word set} % 7
\wordlist{block}

\section{Introduction} % 7.1

\section{Additional terms} % 7.2

\begin{description}
\item[block:]
	1024 characters of data on mass storage,
	designated by a block number.

\item[block buffer:]
	A block-sized region of data space where a block is made
	temporarily available for use. The current block buffer
	is the block buffer most recently accessed by
	\word{BLOCK}, \word{BUFFER}, \word{LOAD}, \word{LIST}, or
	\word{THRU}.
\end{description}

\section{Additional usage requirements} % 7.3

\subsection{Data space} % 7.3.2

A program may access memory within a valid block buffer.

See: \xref[3.3.3 Data Space]{usage:dataspace}.

\subsection{Block buffer regions} % 7.3.3
\label{block:buffers}

The address of a block buffer returned by \word{BLOCK} or
\word{BUFFER} is transient. A call to \word{BLOCK} or \word{BUFFER}
may render a previously-obtained block-buffer address invalid, as
may a call to any word that:
\begin{itemize}
\item parses:
\item displays characters on the user output device, such as
	\word[core]{TYPE} or \word[core]{EMIT};
\item controls the user output device, such as \word[core]{CR} or
	\word[facility]{AT-XY};
\item receives or tests for the presence of characters from the
	user input device such as \word[core]{ACCEPT} or
	\word[core]{KEY};
\item waits for a condition or event, such as \word[facility]{MS}
	or \word[facility]{EKEY};
\item manages the block buffers, such as \word{FLUSH},
	\word{SAVE-BUFFERS}, or \word{EMPTY-BUFFERS};
\item performs any operation on a file or file-name directory
	that implies I/O, such as \word{REFILL} or any word that
	returns an \emph{ior};
\item implicitly performs I/O, such as text interpreter nesting
	and un-nesting when files are being used (including un-nesting
	implied by \word[exception]{THROW}).
\end{itemize}

If the input source is a block, these restrictions also apply to
the address returned by \word[core]{SOURCE}. Block buffers are
uniquely assigned to blocks.

See \xref{rat:block:buffers}.
\subsection{Parsing} % 7.3.4

The Block word set implements an alternative input source for the
text interpreter. When the input source is a block, \word{BLK} shall
contain the non-zero block number and the input buffer is the
1024-character buffer containing that block.

A block is conventionally displayed as 16 lines of 64 characters.

A program may switch the input source to a block by using
\word{LOAD} or \word{THRU}. Input sources may be nested using
\word{LOAD} and \word{EVALUATE} in any order.

A program may reposition the parse area within a block by
manipulating \word[core]{toIN}. More extensive repositioning can be
accomplished using \word[core]{SAVE-INPUT} and
\word[core]{RESTORE-INPUT}.

See: \xref[3.4.1 Parsing]{usage:parsing}.

\subsection{Possible action on an ambiguous condition} % 7.3.5

See: \xref[3.4.4 Possible action on an ambiguous condition]{usage:ambiguous}.

\begin{itemize}
\item A system with the Block word set may set interpretation state
	and interpret a block.
\end{itemize}

\section{Additional documentation requirements} % 7.4

\subsection{System documentation} % 7.4.1

\subsubsection{Implementation-defined options} % 7.4.1.1

\begin{itemize}
\item the format used for display by \wref{block:LIST}{LIST}
	(if implemented);
\item the length of a line affected by \wref{block:bs}{\bs}
	(if implemented).
\end{itemize}

\subsubsection{Ambiguous conditions} % 7.4.1.2

\begin{itemize}
\item Correct block read was not possible;
\item I/O exception in block transfer;
\item Invalid block number (\wref{block:BLOCK}{BLOCK},
	\wref{block:BUFFER}{BUFFER}, \wref{block:LOAD}{LOAD});
\item A program directly alters the contents of
	\wref{block:BLK}{BLK};
\item No current block buffer for \wref{block:UPDATE}{UPDATE}.
\end{itemize}

\subsubsection{Other system documentation} % 7.4.1.3

\begin{itemize}
\item any restrictions a multiprogramming system places on the use
	of buffer addresses;
\item the number of blocks available for source text and data.
\end{itemize}

\subsection{Program documentation} % 7.4.2

\begin{itemize}
\item the number of blocks required by the program.
\end{itemize}


\section{Compliance and labeling} % 7.5

\subsection{Forth-\snapshot{} systems} % 7.5.1

The phrase ``Providing the Block word set'' shall be appended to
the label of any Standard System that provides all of the Block
word set.

The phrase ``Providing \emph{name(s)} from the Block Extensions
word set'' shall be appended to the label of any Standard System
that provides portions of the Block Extensions word set.

The phrase ``Providing the Block Extensions word set'' shall be
appended to the label of any Standard System that provides all of
the Block and Block Extensions word sets.

\subsection{Forth-\snapshot{} programs} % 7.5.2

The phrase ``Requiring the Block word set'' shall be appended to
the label of Standard Programs that require the system to provide
the Block word set.

The phrase ``Requiring \emph{name(s)} from the Block Extensions
word set'' shall be appended to the label of Standard Programs
that require the system to provide portions of the Block Extensions
word set.

The phrase ``Requiring the Block Extensions word set'' shall be
appended to the label of Standard Programs that require the system
to provide all of the Block and Block Extensions word sets.


\section{Glossary} % 7.6

\subsection{Block words} % 7.6.1

\begin{worddef}{0790}{BLK}[b-l-k]
\item \stack{}{a-addr}

	\param{a-addr} is the address of a cell containing zero or the
	number of the mass-storage block being interpreted. If
	\word{BLK} contains zero, the input source is not a block and
	can be identified by \word[core]{SOURCE-ID}, if
	\word[core]{SOURCE-ID} is available. An ambiguous condition
	exists if a program directly alters the contents of \word{BLK}.

\see \xref[7.3.3 Block buffer regions]{block:buffers}.
\end{worddef}


\begin{worddef}{0800}{BLOCK}
\item \stack{u}{a-addr}

	\param{a-addr} is the address of the first character of the
	block buffer assigned to mass-storage block \param{u}. An
	ambiguous condition exists if \param{u} is not an available
	block number.

	If block \param{u} is already in a block buffer, \param{a-addr}
	is the address of that block buffer.

	If block \param{u} is not already in memory and there is an
	unassigned block buffer, transfer block \param{u} from mass
	storage to an unassigned block buffer. \param{a-addr} is the
	address of that block buffer.

	If block \param{u} is not already in memory and there are no
	unassigned block buffers, unassign a block buffer. If the block
	in that buffer has been \word{UPDATE}d, transfer the block to
	mass storage and transfer block \param{u} from mass storage into
	that buffer. \param{a-addr} is the address of that block buffer.

	At the conclusion of the operation, the block buffer pointed to
	by \param{a-addr} is the current block buffer and is assigned to
	\param{u}.
\end{worddef}


\begin{worddef}{0820}{BUFFER}
\item \stack{u}{a-addr}

	\param{a-addr} is the address of the first character of the
	block buffer assigned to block \param{u}. The contents of the
	block are unspecified. An ambiguous condition exists if \param{u}
	is not an available block number.

	If block \param{u} is already in a block buffer, \param{a-addr}
	is the address of that block buffer.

	If block \param{u} is not already in memory and there is an
	unassigned 	buffer, \param{a-addr} is the address of that block
	buffer.

	If block \param{u} is not already in memory and there are no
	unassigned block buffers, unassign a block buffer. If the block
	in that buffer has been \word{UPDATE}d, transfer the block to
	mass storage. \param{a-addr} is the address of that block buffer.

	At the conclusion of the operation, the block buffer pointed to
	by \param{a-addr} is the current block buffer and is assigned to
	\param{u}.

\see \wref{block:BLOCK}{BLOCK}.
\end{worddef}


\begin{worddef}{1360}{EVALUATE}
\item Extend the semantics of \wref{core:EVALUATE}{EVALUATE} to
	include: Store zero in \word{BLK}.
\end{worddef}


\begin{worddef}{1559}{FLUSH}
\item \stack{}{}

	Perform the function of \word{SAVE-BUFFERS}, then unassign all
	block buffers.
\end{worddef}


\begin{worddef}{1790}{LOAD}
\item \stack{i*x u}{j*x}

	Save the current input-source specification. Store \param{u} in
	\word{BLK} (thus making block \param{u} the input source and
	setting the input buffer to encompass its contents), set
	\word[core]{toIN} to zero, and interpret. When the parse area is
	exhausted, restore the prior input source specification. Other
	stack effects are due to the words \word{LOAD}ed.

	An ambiguous condition exists if \param{u} is zero or is not a
	valid block number.

\see \xref[3.4 The Forth text interpreter]{usage:command}.
\end{worddef}

\newpage
\begin{worddef}{2180}{SAVE-BUFFERS}
\item \stack{}{}

	Transfer the contents of each \word{UPDATE}d block buffer to
	mass storage. Mark all buffers as unmodified.
\end{worddef}


\begin{worddef}{2400}{UPDATE}
\item \stack{}{}

	Mark the current block buffer as modified. An ambiguous
	condition exists if there is no current block buffer.

	\word{UPDATE} does not immediately cause I/O.

\see \wref{block:BLOCK}{BLOCK},
	\wref{block:BUFFER}{BUFFER},
	\wref{block:FLUSH}{FLUSH},\\
	\wref{block:SAVE-BUFFERS}{SAVE-BUFFERS}.
\end{worddef}


\subsection{Block extension words} % 7.6.2
\extended

\begin{worddef}{1330}{EMPTY-BUFFERS}
\item \stack{}{}

	Unassign all block buffers. Do not transfer the contents of
	any \word{UPDATE}d block buffer to mass storage.

\see \wref{block:BLOCK}{BLOCK}.
\end{worddef}


\begin{worddef}{1770}{LIST}
\item \stack{u}{}

	Display block \param{u} in an implementation-defined format.
	Store \param{u} in \word{SCR}.

\see \wref{block:BLOCK}{BLOCK}.
\end{worddef}


\begin{worddef}{2125}{REFILL}
\item \stack{}{flag}

	Extend the execution semantics of \wref{core:REFILL}{REFILL}
	with the following:

	When the input source is a block, make the next block the input
	source and current input buffer by adding one to the value of
	\word{BLK} and setting \word[core]{toIN} to zero. Return
	\emph{true} if the new value of BLK is a valid block number,
	otherwise \emph{false}.

\see \wref{core:REFILL}{REFILL},
	\wref{file:REFILL}{REFILL}.
\end{worddef}


\begin{worddef}{2190}{SCR}[s-c-r]
\item \stack{}{a-addr}

	\param{a-addr} is the address of a cell containing the block
	number of the block most recently \word{LIST}ed.

\see \rref{block:SCR}{}.

	\begin{rationale} % A.7.6.2.2190 SCR
		\word{SCR} is short for screen.
	\end{rationale}
\end{worddef}


\begin{worddef}{2280}{THRU}
\item \stack{i*x u_1 u_2}{j*x}

	\word{LOAD} the mass storage blocks numbered \param{u_1} through
	\param{u_2} in sequence. Other stack effects are due to the words
	\word{LOAD}ed.
\end{worddef}


\begin{worddef}[bs]{2535}{\bs}[backslash]
\item Extend the semantics of \wref{core:bs}{\bs} to be:

\compile
	Perform the execution semantics given below.

\execute
	\stack{"ccc<eol>"}{}

	If \word{BLK} contains zero, parse and discard the remainder
	of the parse area; otherwise parse and discard the portion
	of the parse area corresponding to the remainder of the current
	line. \word{bs} is an immediate word.
\end{worddef}
