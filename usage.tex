\chapter{Usage requirements}
\label{usage}

A system shall provide all of the words defined in
\xref[Core Words]{wordlist:core}. It may also provide any
words defined in the optional word sets and extensions word
sets. No standard word provided by a system shall alter the
system state in a way that changes the effect of execution of
any other standard word except as provided in this Standard.
A system may contain non-standard extensions, provided that
they are consistent with the requirements of this Standard.

The implementation of a system may use words and techniques
outside the scope of this Standard.

A system need not provide all words in executable form. The
implementation may provide definitions, including definitions
of words in the Core word set, in source form only. If so,
the mechanism for adding the definitions to the dictionary
is implementation defined.

A program that requires a system to provide words or techniques
not defined in this Standard has an environmental dependency.

\section{Data types} % 3.1
\label{usage:data}

A data type identifies the set of permissible values for a
data object. It is not a property of a particular storage
location or position on a stack. Moving a data object shall
not affect its type.

No data-type checking is required of a system. An ambiguous
condition exists if an incorrectly typed data object is
encountered.

Table \ref{table:datatypes} summarizes the data types used
throughout this Standard. Multiple instances of the same
type in the description of a definition are suffixed with
a sequence digit subscript to distinguish them.

\begin{table}[ht]
  \begin{center}
	\caption{Data types}
	\label{table:datatypes}
	\begin{tabular}{llr}
	\hline\hline
	\emph{Symbol} & \emph{Data type} & \emph{Size on stack} \\
	\hline
	\param{flag}		& flag								& 1 cell \\
	\param{true}		& true flag							& 1 cell \\
	\param{false}		& false flag						& 1 cell \\
	\param{char}		& character							& 1 cell \\
	\param{n}			& signed number						& 1 cell \\
	\param{+n}			& non-negative number				& 1 cell \\
	\param{u}			& unsigned number					& 1 cell \\
	\param{u|n}\footnotemark[1]
						& number							& 1 cell \\
	\param{x}			& unspecified cell					& 1 cell \\
	\param{xt}			& execution token					& 1 cell \\
	\param{addr}		& address							& 1 cell \\
	\param{a-addr}		& aligned address					& 1 cell \\
	\param{c-addr}		& character-aligned address			& 1 cell \\
	\param{d}			& double-cell signed number			& 2 cells \\
	\param{+d}			& double-cell non-negative number	& 2 cells \\
	\param{ud}			& double-cell unsigned number		& 2 cells \\
	\param{d|ud}\footnotemark[2]
						& double-cell number				& 2 cells \\
	\param{xd}			& unspecified cell pair				& 2 cells \\
	\param{colon-sys}	& definition compilation			& implementation dependent \\
	\param{do-sys}		& do-loop structures				& implementation dependent \\
	\param{case-sys}	& \word{CASE} structures			& implementation dependent \\
	\param{of-sys}		& \word{OF} structures				& implementation dependent \\
	\param{orig}		& control-flow origins				& implementation dependent \\
	\param{dest}		& control-flow destinations			& implementation dependent \\
	\param{loop-sys}	& loop-control parameters			& implementation dependent \\
	\param{nest-sys}	& definition cells					& implementation dependent \\
	\param{i*x, j*x, k*x}\footnotemark[3]
						& any data type						& 0 or more cells \\
	\hline\hline
	\end{tabular}
\par
	\begin{tabular}{lp{0.8\textwidth}}
	\footnotemark[1] &
		May be either a signed number or an unsigned number
		depending on context. \\
	\footnotemark[2] &
		May be either a double-cell signed number or a double-cell
		unsigned number depending on context.\\
	\footnotemark[3] &
		May be an undetermined number of stack entries of
		unspecified type. For examples of use, see
		\wref{core:EXECUTE}{EXECUTE}, \wref{core:QUIT}{QUIT}.
	\end{tabular}
  \end{center}
\end{table}

\subsection{Data-type relationships} % 3.1.1

Some of the data types are subtypes of other data types. A data
type \param{i} is a subtype of type \param{j} if and only if the
members of \param{i} are a subset of the members of \param{j}. The
following list represents the subtype relationships using the
phrase ``\param{i} $\Rightarrow$ \param{j}'' to denote ``\param{i}
is a subtype of \param{j}''. The subtype relationship is transitive;
if \param{i} $\Rightarrow$ \param{j} and \param{j} $\Rightarrow$
\param{k} then \param{i} $\Rightarrow$ \param{k}:

\begin{quote}
\param{+n} $\Rightarrow$ \param{u} $\Rightarrow$ \param{x}; \\
\param{+n} $\Rightarrow$ \param{n} $\Rightarrow$ \param{x}; \\
\param{char} $\Rightarrow$ \param{+n}; \\
\param{a-addr} $\Rightarrow$ \param{c-addr}
	$\Rightarrow$ \param{addr}
	$\Rightarrow$ \param{u}; \\
\param{flag} $\Rightarrow$ \param{x}; \\
\param{xt} $\Rightarrow$ \param{x}; \\
\param{+d} $\Rightarrow$ \param{d} $\Rightarrow$ \param{xd}; \\
\param{+d} $\Rightarrow$ \param{ud} $\Rightarrow$ \param{xd}.
\end{quote}

Any Forth definition that accepts an argument of type \param{i}
shall also accept an argument that is a subtype of \param{i}.


\subsection{Character types} % 3.1.2
\label{usage:char}

Characters shall be at least one address unit wide, contain
at least eight bits, and have a size less than or equal to
cell size.

The characters provided by a system shall include the graphic
characters \{32 {\ldots} 126\}, which represent graphic forms
as shown in table \ref{table:ASCII}.

\subsubsection{Graphic characters} % 3.1.2.1
\label{usage:ASCII}

A graphic character is one that is normally displayed (e.g.,
A, \#, \&, 6). These values and graphics, shown in table
\ref{table:ASCII}, are taken directly from ANS X3.4-1974 (ASCII)
and ISO 646-1983, International Reference Version (IRV). The
graphic forms of characters outside the hex range \{20 {\ldots}
7E\} are implementation defined. Programs that use the graphic hex
24 (the currency sign) have an environmental dependency.

The graphic representation of characters is not restricted to
particular type fonts or styles. The graphics here are examples.

\subsubsection{Control characters} % 3.1.2.2
\label{usage:control}

All non-graphic characters included in the implementation-defined
character set are defined in this Standard as control characters.
In particular, the characters \{0 {\ldots} 31\}, which could be
included in the implementation-defined character set, are control
characters.

Programs that require the ability to send or receive control
characters have an environmental dependency.

\begin{table}[ht]
  \begin{center}
	\caption{Standard graphic characters}
	\label{table:ASCII}
	\small
	\begin{tabular}{cc@{}c|cc@{}c|cc@{}c|cc@{}c|cc@{}c|cc@{}c}
	\hline\hline
	Hex & IRV && Hex & IRV && Hex & IRV && Hex & IRV && Hex & IRV && Hex & IRV \\
	\multicolumn{3}{l|}{ASCII} &
	\multicolumn{3}{l|}{ASCII} &
	\multicolumn{3}{l|}{ASCII} &
	\multicolumn{3}{l|}{ASCII} &
	\multicolumn{3}{l|}{ASCII} &
	\multicolumn{3}{l}{ASCII} \\
	\hline
	20&   &		&30& 0 & 0	&40& @ & @	&50& P & P	&60& \verb|`| & \verb|`|
																&70& p & p	\\
	21& ! & !	&31& 1 & 1	&41& A & A	&51& Q & Q	&61& a & a	&71& q & q	\\
	22& \verb|"| & \verb|"|
				&32& 2 & 2	&42& B & B	&52& R & R	&62& b & b	&72& r & r	\\
	23& \# & \# &33& 3 & 3	&43& C & C	&53& S & S	&63& c & c	&73& s & s	\\
	24& \textcurrency  & \$ &34& 4 & 4	&44& D & D	&54& T & T	&63& d & d	&74& t & t	\\
	25& \% & \% &35& 5 & 5	&45& E & E	&55& U & U	&64& e & e	&75& u & u	\\
	26& \& & \& &36& 6 & 6	&46& F & F	&56& V & V	&65& f & f	&76& v & v	\\
	27& '  & '	&37& 7 & 7	&47& G & G	&57& W & W	&66& g & g	&77& w & w	\\
	28& (  & (	&38& 8 & 8	&48& H & H	&58& X & X	&67& h & h	&78& x & x	\\
	29& )  & )	&39& 9 & 9	&49& I & I	&59& Y & Y	&68& i & i	&79& y & y	\\
	2A& *  & *	&3A& : & :	&4A& J & J	&5A& Z & Z	&69& j & j	&7A& z & z	\\
	2B& +  & +	&3B& ; & ;	&4B& K & K	&5B& [ & [	&6A& k & k	&7B& \{&\{	\\
	2C& ,  & ,	&3C&$<$&$<$	&4C& L & L	&5C& \verb|\| & \verb|\|
													&6C& l & l	&7C& \verb"|" & \verb"|" \\
	2D& -  & -	&3D& = & =	&4D& M & M	&5D& ] & ]	&6D& m & m	&7D& \}&\}	\\
	2E& .  & .	&3E&$>$&$>$	&4E& N & N	&5E& \verb|^| & \verb|^|
													&6E& n & n	&7E& \verb|~| & \verb|~| \\
	2F& /  & /	&3F& ? & ?	&4F& O & O	&5F& \_ & \_&6F& o & o	\\
	\hline\hline
	\end{tabular}
  \end{center}
\end{table}

\subsection{Single-cell types} % 3.1.3
\label{usage:cell}

The implementation-defined fixed size of a cell is specified in
address units and the corresponding number of bits.
See \xref[E.2 Hardware peculiarities]{port:hardware}.

Cells shall be at least one address unit wide and contain at least
sixteen bits. The size of a cell shall be an integral multiple of
the size of a character. Data-stack elements, return-stack elements,
addresses, execution tokens, flags, and integers are one cell wide.

\subsubsection{Flags} % 3.1.3.1
\label{usage:flags}

Flags may have one of two logical states, \emph{true} or \emph{false}.
Programs that use flags as arithmetic operands have an environmental
dependency. A true flag returned by a standard word shall be a
single-cell value with all bits set. A false flag returned by a
standard word shall be a single-cell value with all bits clear.

\subsubsection{Integers} % 3.1.3.2

The implementation-defined range of signed integers shall include
\{-32767 {\ldots} +32767\}. The imple\-mentation-defined range of
non-negative integers shall include \{0 {\ldots} 32767\}. The
imple\-mentation-defined range of unsigned integers shall include
\{0 {\ldots} 65535\}.

\subsubsection{Addresses} % 3.1.3.3
\label{usage:addr}

An address identifies a location in data space with a size of one
address unit, which a program may fetch from or store into except
for the restrictions established in this Standard. The size of an
address unit is specified in bits. Each distinct address value
identifies exactly one such storage element.
See \xref[3.3.3 Data space]{usage:dataspace}.

The set of character-aligned addresses, addresses at which a
character can be accessed, is an implementation-defined subset of
all addresses. Adding the size of a character to a character-aligned
address shall produce another character-aligned address.

The set of aligned addresses is an implementation-defined subset
of character-aligned addresses. Adding the size of a cell to an
aligned address shall produce another aligned address.

\subsubsection{Counted strings} % 3.1.3.4
\label{usage:cstring}

A counted string in memory is identified by the address
(\emph{c-addr}) of its length character.

The length character of a counted string shall contain a binary
representation of the number of data characters, between zero and
the implementation-defined maximum length for a counted string.
The maximum length of a counted string shall be at least 255.

\subsubsection{Execution tokens} % 3.1.3.5

Different definitions may have the same execution token if the
definitions are equivalent.

\subsection{Cell-pair types} % 3.1.4
\label{usage:2cell}

A cell pair in memory consists of a sequence of two contiguous
cells. The cell at the lower address is the first cell, and its
address is used to identify the cell pair. Unless otherwise
specified, a cell pair on a stack consists of the first cell
immediately above the second cell.

\subsubsection{Double-cell integers} % 3.1.4.1

On the stack, the cell containing the most significant part of a
double-cell integer shall be above the cell containing the least
significant part.

The implementation-defined range of double-cell signed integers
shall include \{-2147483647 {\ldots} +2147483647\}.

The implementation-defined range of double-cell non-negative
integers shall include \{0 {\ldots} 2147483647\}.

The implementation-defined range of double-cell unsigned integers
shall include \{0 {\ldots} 4294967295\}. Placing the single-cell
integer zero on the stack above a single-cell unsigned integer
produces a double-cell unsigned integer with the same value.
See \xref[Internal number representation]{usage:number}.

\subsubsection{Character strings} % 3.1.4.2

A string is specified by a cell pair (\emph{c-addr u}) representing
its starting address and length in characters.

\subsection{System types} % 3.1.5

The system data types specify permitted word combinations during
compilation and execution.

\subsubsection{System-compilation types} % 3.1.5.1

These data types denote zero or more items on the control-flow stack
(see \ref{usage:controlstack}). The possible presence of such items
on the data stack means that any items already there shall be
unavailable to a program until the control-flow-stack items are
consumed.

The implementation-dependent data generated upon beginning to compile
a definition and consumed at its close is represented by the symbol
\emph{colon-sys} throughout this Standard.

The implementation-dependent data generated upon beginning to
compile a do-loop structure such as \word{DO} {\ldots} \word{LOOP}
and consumed at its close is represented by the symbol \emph{do-sys}
throughout this Standard.

The implementation-dependent data generated upon beginning to
compile a \word{CASE} {\ldots} \word{ENDCASE} structure and consumed
at its close is represented by the symbol \emph{case-sys} throughout
this Standard.

The implementation-dependent data generated upon beginning to
compile an \word{OF} {\ldots} \word{ENDOF} structure and consumed
at its close is represented by the symbol \emph{of-sys} throughout
this Standard.

The implementation-dependent data generated and consumed by executing
the other standard control-flow words is represented by the symbols
\emph{orig} and \emph{dest} throughout this Standard.

\subsubsection{System-execution types} % 3.1.5.2

These data types denote zero or more items on the return stack.
Their possible presence means that any items already on the return
stack shall be unavailable to a program until the system-execution
items are consumed.

The implementation-dependent data generated upon beginning to
execute a definition and consumed upon exiting it is represented
by the symbol \emph{nest-sys} throughout this Standard.

The implementation-dependent loop-control parameters used to
control the execution of do-loops are represented by the symbol
\emph{loop-sys} throughout this Standard. Loop-control parameters
shall be available inside the do-loop for words that use or change
these parameters, words such as \word{I}, \word{J}, \word{LEAVE}
and \word{UNLOOP}.


\section{The implementation environment} % 3.2 =======================

\subsection{Numbers} % 3.2.1

\subsubsection{Internal number representation} % 3.2.1.1
\label{usage:number}

This Standard allows one's complement, two's complement, or
sign-magnitude number representations and arithmetic. Arithmetic
zero is represented as the value of a single cell with all bits
clear.

The representation of a number as a compiled literal or in memory
is implementation dependent.

\subsubsection{Digit conversion} % 3.2.1.2
\label{usage:digits}

Numbers shall be represented externally by using characters from
the standard character set. Conversion between the internal and
external forms of a digit shall behave as follows:

The value in \word{BASE} is the radix for number conversion. A
digit has a value ranging from zero to one less than the contents
of \word{BASE}. The digit with the value zero corresponds to the
character ``0''. This representation of digits proceeds through
the character set to the decimal value nine corresponding to the
character ``9''. For digits beginning with the decimal value ten
the graphic characters beginning with the character ``A'' are used.
This correspondence continues up to and including the digit with
the decimal value thirty-five which is represented by the character
``Z''. The conversion of digits outside this range is implementation
defined.

\subsubsection{Free-field number display} % 3.2.1.3
\label{usage:dot}

Free-field number display uses the characters described in digit
conversion, without leading zeros, in a field the exact size of
the converted string plus a trailing space. If a number is zero,
the least significant digit is not considered a leading zero. If
the number is negative, a leading minus sign is displayed.

Number display may use the pictured numeric output string buffer
to hold partially converted strings (see \xref[Other transient
regions]{usage:transient}).

\subsection{Arithmetic} % 3.2.2

\subsubsection{Integer division} % 3.2.2.1
\label{usage:div}

Division produces a quotient \emph{q} and a remainder \emph{r}
by dividing operand \emph{a} by operand \emph{b}. Division
operations return \emph{q, r}, or both. The identity
$b \times q + r = a$ shall hold for all \emph{a} and \emph{b}.

When unsigned integers are divided and the remainder is not zero,
\emph{q} is the largest integer less than the true quotient.

When signed integers are divided, the remainder is not zero, and
\emph{a} and \emph{b} have the same sign, \emph{q} is the largest
integer less than the true quotient. If only one operand is
negative, whether \emph{q} is rounded toward negative infinity
(floored division) or rounded towards zero (symmetric division) is
implementation defined.

Floored division is integer division in which the remainder carries
the sign of the divisor or is zero, and the quotient is rounded to
its arithmetic floor. Symmetric division is integer division in
which the remainder carries the sign of the dividend or is zero and
the quotient is the mathematical quotient ``rounded towards zero''
or ``truncated''. Examples of each are shown in tables
\ref{table:floor} and \ref{table:round}.

In cases where the operands differ in sign and the rounding
direction matters, a program shall either include code generating
the desired form of division, not relying on the
implementation-defined default result, or have an environmental
dependency on the desired rounding direction.

\begin{table}[ht]
  \begin{minipage}{0.5\textwidth}
	\begin{center}
		\caption{Floored Division Example}

		\label{table:floor}
		\begin{tabular}{lrllrllrllrl}
		\hline\hline
		\multicolumn{3}{c}{Dividend} &
		\multicolumn{3}{c}{Divisor} &
		\multicolumn{3}{c}{Remainder} &
		\multicolumn{3}{c}{Quotient} \\
		\hline
		&  10 &&&  7 &&&  3 &&&  1 \\
		& -10 &&&  7 &&&  4 &&& -2 \\
		&  10 &&& -7 &&& -4 &&& -2 \\
		& -10 &&& -7 &&& -3 &&&  1 \\
		\hline\hline
		\end{tabular}
	\end{center}
  \end{minipage}
  \begin{minipage}{0.5\textwidth}
	\begin{center}
		\caption{Symmetric Division Example}
		\label{table:round}
		\begin{tabular}{lrllrllrllrl}
		\hline\hline
		\multicolumn{3}{c}{Dividend} &
		\multicolumn{3}{c}{Divisor} &
		\multicolumn{3}{c}{Remainder} &
		\multicolumn{3}{c}{Quotient} \\
		\hline
		&  10 &&&  7 &&&  3 &&&  1 \\
		& -10 &&&  7 &&& -3 &&& -1 \\
		&  10 &&& -7 &&&  3 &&& -1 \\
		& -10 &&& -7 &&& -3 &&&  1 \\
		\hline\hline
		\end{tabular}
	\end{center}
  \end{minipage}
\end{table}

\subsubsection{Other integer operations} % 3.2.2.2
\label{usage:intops}

In all integer arithmetic operations, both overflow and underflow
shall be ignored. The value returned when either overflow or
underflow occurs is implementation defined.

\subsection{Stacks} % 3.2.3

\subsubsection{Data stack} % 3.2.3.1
\label{usage:datastack}

Objects on the data stack shall be one cell wide.

\subsubsection{Control-flow stack} % 3.2.3.2
\label{usage:controlstack}

The control-flow stack is a last-in, first out list whose elements
define the permissible matchings of control-flow words and the
restrictions imposed on data-stack usage during the compilation of
control structures.

The elements of the control-flow stack are system-compilation data
types.

The control-flow stack may, but need not, physically exist in an
implementation. If it does exist, it may be, but need not be,
implemented using the data stack. The format of the control-flow
stack is implementation defined. Since the control-flow stack may
be implemented using the data stack, items placed on the data stack
are unavailable to a program after items are placed on the
control-flow stack and remain unavailable until the control-flow
stack items are removed.

\subsubsection{Return stack} % 3.2.3.3
\label{usage:returnstack}

Items on the return stack shall consist of one or more cells. A
system may use the return stack in an implementation-dependent
manner during the compilation of definitions, during the execution
of do-loops, and for storing run-time nesting information.

A program may use the return stack for temporary storage during the
execution of a definition subject to the following restrictions:

\begin{itemize}
\item A program shall not access values on the return stack
	(using \word{R@}, \word{Rfrom}, \word{2R@} or \word{2Rfrom})
	that it did not place there using \word{toR} or \word{2toR};

\item A program shall not access from within a do-loop values
	placed on the return stack before the loop was entered;

\item All values placed on the return stack within a do-loop
	shall be removed before \word{I}, \word{J}, \word{LOOP},
	\word{+LOOP}, \word{UNLOOP}, or \word{LEAVE} is executed;

\item All values placed on the return stack within a definition
	shall be removed before the definition is terminated or
	before \word{EXIT} is executed.
\end{itemize}

\subsection{Operator terminal} % 3.2.4

See \xref[Exclusions]{intro:exclusions}.

\subsubsection{User input device} % 3.2.4.1
\label{usage:input}

The method of selecting the user input device is implementation
defined.

The method of indicating the end of an input line of text is
implementation defined.

\subsubsection{User output device} % 3.2.4.2
\label{usage:output}

The method of selecting the user output device is implementation
defined.

\subsection{Mass storage} % 3.2.5

A system need not provide any standard words for accessing mass
storage. If a system provides any standard word for accessing
mass storage, it shall also implement the Block word set.

\subsection{Environmental queries} % 3.2.6
\label{usage:env}

The name spaces for \word{ENVIRONMENTq} and definitions are
disjoint. Names of definitions that are the same as
\word{ENVIRONMENTq} strings shall not impair the operation of
\word{ENVIRONMENTq}. Table \ref{table:env} contains
the valid input strings and corresponding returned value for
inquiring about the programming environment with
\word{ENVIRONMENTq}.

\begin{table}[ht]
  \begin{center}
	\caption{Environmental Query Strings}
	\label{table:env}
	\begin{tabular}{p{9.5em}rcp{0.42\textwidth}}
		\hline\hline
		\multicolumn{2}{l}{String \hfill Value data type} & Constant? & Meaning \\
		\hline
		\texttt{/COUNTED-STRING}	& \emph{n}		& yes	&
			maximum size of a counted string, in characters \\
		\texttt{/HOLD}				& \emph{n}		& yes	&
			size of the pictured numeric output string buffer,

			in characters \\

		\texttt{/PAD}				& \emph{n}		& yes	&
			size of the scratch area pointed to by \word{PAD},
			in characters \\
		\texttt{ADDRESS-UNIT-BITS}	& \emph{n}		& yes	&
			size of one address unit, in bits \\
		\texttt{CORE}				& \emph{flag}	& no	&
			true if complete core word set present
			(i.e., not a subset as defined in \ref{label:system}) \\
		\texttt{CORE-EXT}			& \emph{flag}	& no	&
			true if core extensions word set present \\
		\texttt{FLOORED}			& \emph{flag}	& yes	&
			true if floored division is the default \\
		\texttt{MAX-CHAR}			& \emph{u}		& yes	&
			maximum value of any character in the
			imple\-mentation-defined character set \\
		\texttt{MAX-D}				& \emph{d}		& yes	&
			largest usable signed double number \\
		\texttt{MAX-N}				& \emph{n}		& yes	&
			largest usable signed integer \\
		\texttt{MAX-U}				& \emph{u}		& yes	&
			largest usable unsigned integer \\
		\texttt{MAX-UD}				& \emph{ud}		& yes	&
			largest usable unsigned double number \\
		\texttt{RETURN-STACK-CELLS}	& \emph{n}		& yes	&
			maximum size of the return stack, in cells \\
		\texttt{STACK-CELLS}		& \emph{n}		& yes	&
			maximum size of the data stack, in cells \\
		\hline\hline
	\end{tabular}
  \end{center}
\end{table}

If an environmental query (using \word{ENVIRONMENTq}) returns
\emph{false} (i.e., unknown) in response to a string, subsequent
queries using the same string may return \emph{true}. If a query
returns \emph{true} (i.e., known) in response to a string,
subsequent queries with the same string shall also return
\emph{true}. If a query designated as constant in the above table
returns \emph{true} and a value in response to a string,
subsequent queries with the same string shall return \emph{true}
and the same value.


\subsection{Extension queries} % 3.2.7
\label{usage:extensions}
~ \hfill \textsf{\small X:extension-query}

As part of the Forth 200\emph{x} standards procedure, additions to the
Standard \replace{ed07}{where}{are} labelled as \emph{extensions}. These extensions have
been added to the environmental query name space. \word{ENVIRONMENTq}
a \emph{true} if the system has implemented the extension as
documented. Table \ref{table:extensions} contains the valid input
strings corresponding to the documented extensions. In order to
distinguish such extensions, they start with the string
``\textsf{X:}''.

The extension to the environment query table (\ref{usage:env}) is
itself an extension. Known as the \textsf{X:extension-query} extension.

\begin{table}[ht]
  \begin{center}
	\caption{Forth 200\emph{x} Extensions}
	\label{table:extensions}
	\begin{tabular}{p{9.5em}rcp{0.42\textwidth}}
		\hline\hline
		\multicolumn{2}{l}{String \hfill Value data type} & Constant? & Meaning \\
		\hline
		\texttt{X:deferred}			& -- & -- & the \textsf{X:deferred} extension is present \\
	\place{x:defined}{\texttt{X:defined}}		& -- & -- & \place{x:defined}{the \textsf{X:defined} extension is present} \\
	\place{x:ekeys}{\texttt{X:ekeys}}			& -- & -- & \place{x:ekeys}{the \textsf{X:ekeys} extension is present} \\
		\texttt{X:extension-query}  & -- & -- & the \textsf{X:extension-query} extension is present\\
	\place{x:fp-stack}{\texttt{X:fp-stack}} & -- & -- & \place{x:fp-stack}{\textsf{X;fp-stack} extension is present} \\
	\place{x:number-prefix}{\texttt{X:number-prefix}} & -- & -- & \place{x:number-prefix}{\textsf{X:number-prefix} extension is present} \\
		\texttt{X:parse-name}		& -- & -- & the \textsf{X:parse-name} extension is present \\
	\place{x:required}{\texttt{X:required}}	& -- & -- & \place{x:required}{the \textsf{X:required} extension is present} \\
	\place{x:structures}{\texttt{X:structures}} & -- & -- & \place{x:structures}{the \textsf{X:structures} extension is present} \\
	\place{x:thorw-iors}{\texttt{X:throw-iors}} & -- & -- & \place{x:throw-iors}{the \textsf{X:throw-iors} extension is present} \\
		\hline\hline
	\end{tabular}
  \end{center}
\end{table}

\section{The Forth dictionary} % 3.3 ================================
\label{usage:dict}

Forth words are organized into a structure called the dictionary.
While the form of this structure is not specified by the Standard,
it can be described as consisting of three logical parts:
a name space, a code space, and a data space. The logical separation
of these parts does not require their physical separation.

A program shall not fetch from or store into locations outside data
space. An ambiguous condition exists if a program addresses name
space or code space.

\subsection{Name space} % 3.3.1

The relationship between name space and data space is implementation
dependent.

\subsubsection{Word lists} % 3.3.1.1

The structure of a word list is implementation dependent. When
duplicate names exist in a word list, the latest-defined duplicate
shall be the one found during a search for the name.

\subsubsection{Definition names} % 3.3.1.2
\label{usage:names}

Definition names shall contain \{1 {\ldots} 31\} characters. A
system may allow or prohibit the creation of definition names
containing non-standard characters.

Programs that use lower case for standard definition names or depend
on the case-sensitivity properties of a system have an environmental
dependency.

A program shall not create definition names containing non-graphic
characters.

\subsection{Code space} % 3.3.2

The relationship between code space and data space is implementation
dependent.

\subsection{Data space} % 3.3.3
\label{usage:dataspace}

Data space is the only logical area of the dictionary for which
standard words are provided to allocate and access regions of
memory. These regions are: contiguous regions, variables,
text-literal regions, input buffers, and other transient regions,
each of which is described in the following sections. A program may
read from or write into these regions unless otherwise specified.

\subsubsection{Address alignment} % 3.3.3.1
\label{usage:aaddr}

Most addresses used in ANS Forth are aligned addresses (indicated
by \emph{a-addr}) or character-aligned (indicated by \emph{c-addr}).
\word{ALIGNED}, \word{CHAR+}, and arithmetic operations can alter
the alignment state of an address on the stack. \word{CHAR+} applied
to an aligned address returns a character-aligned address that can
only be used to access characters. Applying \word{CHAR+} to a
character-aligned address produces the succeeding character-aligned
address. Adding or subtracting an arbitrary number to an address can
produce an unaligned address that shall not be used to fetch or
store anything. The only way to find the next aligned address is
with \word{ALIGNED}. An ambiguous condition exists when
\word{@}, \word{!}, \word{,} (comma), \word{+!}, \word{2@}, or
\word{2!} is used with an address that is not aligned, or when
\word{C@}, \word{C!}, or \word{C,} is used with an address that is
not character-aligned.

The definitions of \wref{core:CREATE}{CREATE} and
\wref{core:VARIABLE}{VARIABLE} require that the definitions created
by them return aligned addresses.

After definitions are compiled or the word \word{ALIGN} is executed
the data-space pointer is guaranteed to be aligned.

\subsubsection{Contiguous regions} % 3.3.3.2
\label{usage:contiguous}

A system guarantees that a region of data space allocated using
\word{ALLOT}, \word{,} (comma), \word{C,} (c-comma), and
\word{ALIGN} shall be contiguous with the last region allocated
with one of the above words, unless the restrictions in the
following paragraphs apply. The data-space pointer \word{HERE}
always identifies the beginning of the next data-space region to be
allocated. As successive allocations are made, the data-space
pointer increases. A program may perform address arithmetic within
contiguously allocated regions. The last region of data space
allocated using the above operators may be released by allocating a
corresponding negatively-sized region using \word{ALLOT}, subject
to the restrictions of the following paragraphs.

\word{CREATE} establishes the beginning of a contiguous region of
data space, whose starting address is returned by the \word{CREATE}d
definition. This region is terminated by compiling the next
definition.

Since an implementation is free to allocate data space for use by
code, the above operators need not produce contiguous regions of
data space if definitions are added to or removed from the
dictionary between allocations. An ambiguous condition exists if
deallocated memory contains definitions.


\subsubsection{Variables} % 3.3.3.3
\label{usage:var}

The region allocated for a variable may be non-contiguous with
regions subsequently allocated with \word{,} (comma) or
\word{ALLOT}. For example, in:
\begin{quote}
	\word{VARIABLE} X 1 \word{CELLS} \word{ALLOT}
\end{quote}
the region \texttt{X} and the region \word{ALLOT}ted could be
non-contiguous.

Some system-provided variables, such as \word{STATE}, are
restricted to read-only access.


\subsubsection{Text-literal regions} % 3.3.3.4
\label{usage:"literal}

The text-literal regions, specified by strings compiled with
\word{Sq} and \word{Cq}, may be read-only.

A program shall not store into the text-literal regions created
by \word{Sq} and \word{Cq} nor into any read-only system variable
or read-only transient regions. An ambiguous condition exists when
a program attempts to store into read-only regions.

\subsubsection{Input buffers} % 3.3.3.5
\label{usage:inbuf}

The address, length, and content of the input buffer may be
transient. A program shall not write into the input buffer. In the
absence of any optional word sets providing alternative input
sources, the input buffer is either the terminal-input buffer, used
by \word{QUIT} to hold one line from the user input device, or a
buffer specified by \word{EVALUATE}. In all cases, \word{SOURCE}
returns the beginning address and length in characters of the
current input buffer.

The minimum size of the terminal-input buffer shall be 80
characters.

The address and length returned by \word{SOURCE}, the string
returned by \word{PARSE}, and directly computed input-buffer
addresses are valid only until the text interpreter does I/O to
refill the input buffer or the input source is changed.

A program may modify the size of the parse area by changing the
contents of \word{toIN} within the limits imposed by this Standard.
For example, if the contents of \word{toIN} are saved before a
parsing operation and restored afterwards, the text that was parsed
will be available again for subsequent parsing operations. The
extent of permissible repositioning using this method depends on the
input source (see \xref[(7.3.3) Block buffer regions]{block:buffers}
and \xref[(11.3.4) Input source]{file:source}).

A program may directly examine the input buffer using its address
and length as returned by \word{SOURCE}; the beginning of the parse
area within the input buffer is indexed by the number in \word{toIN}.
The values are valid for a limited time. An ambiguous condition
exists if a program modifies the contents of the input buffer.

\subsubsection{Other transient regions} % 3.3.3.6
\label{usage:transient}

The data space regions identified by \word{PAD}, \word{WORD}, and
\word{num-end} (the pictured numeric output string buffer) may be
transient. Their addresses and contents may become invalid after:

\begin{itemize}
\item a definition is created via a defining word;
\item definitions are compiled with \word{:} or \word{:NONAME};
\item data space is allocated using \word{ALLOT}, \word{,} (comma),
	\word{C,} (c-comma), or \word{ALIGN}.
\end{itemize}

The previous contents of the regions identified by \word{WORD} and
\word{num-end} may be invalid after each use of these words. Further,
the regions returned by \word{WORD} and \word{num-end} may overlap in
memory. Consequently, use of one of these words can corrupt a region
returned earlier by a different word. The other words that construct
pictured numeric output strings (\word{num-start}, \word{num}, \word{numS},
and \word{HOLD})  may also modify the contents of these regions.
Words that display numbers may be implemented using pictured numeric
output words. Consequently, \word{d} (dot), \word{.R},
\word[tools]{.S}, \word[tools]{q}, \word[double]{D.},
\word[double]{D.R}, \word{Ud}, and \word{U.R} could also corrupt
the regions.

The size of the scratch area whose address is returned by \word{PAD}
shall be at least 84 characters. The contents of the region
addressed by \word{PAD} are intended to be under the complete
control of the user: no words defined in this Standard place
anything in the region, although changing data-space allocations as
described in \xref[Contiguous regions]{usage:contiguous} may change
the address returned by \word{PAD}. Non-standard words provided by
an implementation may use \word{PAD}, but such use shall be
documented.

The size of the region identified by \word{WORD} shall be at least
33 characters.

The size of the pictured numeric output string buffer shall be at
least $(2 \times n) + 2 $ characters, where $n$ is the number of
bits in a cell. Programs that consider it a fixed area with
unchanging access parameters have an environmental dependency.

\section{The Forth text interpreter} % 3.4 ==========================
\label{usage:command}

Upon start-up, a system shall be able to interpret, as described
by \wref{core:QUIT}{QUIT}, Forth source code received interactively
from a user input device.

Such interactive systems usually furnish a ``prompt'' indicating
that they have accepted a user request and acted on it. The
imple\-mentation-defined Forth prompt should contain the word ``OK''
in some combination of upper or lower case.

Text interpretation (see \wref{core:EVALUATE}{EVALUATE} and
\wref{core:QUIT}{QUIT}) shall repeat the following steps until
either the parse area is empty or an ambiguous condition exists:

\begin{enumerate}
\item Skip leading spaces and parse a \emph{name}
	(see \ref{usage:parsing});

\item Search the dictionary name space (see \ref{usage:find}).
	If a definition name matching the string is found:

	\begin{enumerate}
	\item if interpreting, perform the interpretation semantics of
		the definition (see \ref{usage:interpret}), and continue at
		a).

	\item if compiling, perform the compilation semantics of the
		definition (see \ref{usage:compile}), and continue at a).
	\end{enumerate}

\item If a definition name matching the string is not found,
	attempt to convert the string to a number
	(see \ref{usage:numbers}). If successful:
	\begin{enumerate}
	\item if interpreting, place the number on the data stack,
		and continue at a);
	\item if compiling, compile code that when executed will place
		the number on the stack (see \wref{core:LITERAL}{LITERAL}),
		and continue at a);
	\end{enumerate}

\item If unsuccessful, an ambiguous condition exists
	(see \ref{usage:ambiguous}).
\end{enumerate}

\subsection{Parsing} % 3.4.1 Parsing
\label{usage:parsing}

Unless otherwise noted, the number of characters parsed may be from
zero to the implementation-defined maximum length of a counted
string.

If the parse area is empty, i.e., when the number in \word{toIN} is
equal to the length of the input buffer, or contains no characters
other than delimiters, the selected string is empty. Otherwise, the
selected string begins with the next character in the parse area,
which is the character indexed by the contents of \word{toIN}. An
ambiguous condition exists if the number in \word{toIN} is greater
than the size of the input buffer.

If delimiter characters are present in the parse area after the
beginning of the selected string, the string continues up to and
including the character just before the first such delimiter, and
the number in \word{toIN} is changed to index immediately past that
delimiter, thus removing the parsed characters and the delimiter
from the parse area. Otherwise, the string continues up to and
including the last character in the parse area, and the number in
\word{toIN} is changed to the length of the input buffer, thus
emptying the parse area.

Parsing may change the contents of \word{toIN}, but shall not affect
the contents of the input buffer. Specifically, if the value in
\word{toIN} is saved before starting the parse, resetting \word{toIN}
to that value immediately after the parse shall restore the parse
area without loss of data.

\subsubsection{Delimiters} % 3.4.1.1
\label{usage:delim}

If the delimiter is the space character, hex 20 (\word{BL}), control
characters may be treated as delimiters. The set of conditions, if
any, under which a ``space'' delimiter matches control characters is
implementation defined.

To skip leading delimiters is to pass by zero or more contiguous
delimiters in the parse area before parsing.

\subsubsection{Syntax} % 3.4.1.2
\label{usage:syntax}

Forth has a simple, operator-ordered syntax. The phrase
\texttt{A B C} returns values as if \texttt{A} were executed first,
then \texttt{B} and finally \texttt{C}. Words that cause deviations
from this linear flow of control are called control-flow words.
Combinations of control-flow words whose stack effects are
compatible form control-flow structures. Examples of typical use are
given for each control-flow word in \ref{annex:rationale} (Annex A).

Forth syntax is extensible; for example, new control-flow words can
be defined in terms of existing ones. This Standard does not require
a syntax or program-construct checker.


\subsubsection{Text interpreter input number conversion} % 3.4.1.3
\label{usage:numbers}

\cbstart\marginpar{\tiny x:number-prefix}
\sout{% Remove
When converting input numbers, the text interpreter shall recognize
both positive and negative numbers, with a negative number
represented by a single minus sign, the character ``\texttt{-}'',
preceding the digits. The value in \word{BASE} is the radix for
number conversion.}

\uline{% place
When converting input numbers, the text interpreter shall recognize
integer numbers in the form \arg{BASEnum}; if the
\textsf{X:number-prefixes} extension is present, the text
interpreter shall recognize integer numbers in the form
\arg{anynum}.}

\begin{center}
\begin{tabular}{r@{ \uline{\textsf{:=}} }l}
\uline{\arg{anynum}}	& \uline{\{ \arg{BASEnum}
		{\textbar} \arg{decnum}
		{\textbar} \arg{hexnum}
		{\textbar} \arg{binnum}
		{\textbar} \arg{cnum} \}} \\
\uline{\arg{BASEnum}}	& \uline{[\textbf{-}]\arg{bdigit}\arg{bdigit}*} \\
\uline{\arg{decnum}}	& \uline{\textbf{\#}[\textbf{-}]\arg{decdigit}\arg{decdigit}*} \\
\uline{\arg{hexnum}}	& \uline{\textbf{\$}[\textbf{-}]\arg{hexdigit}\arg{hexdigit}*} \\
\uline{\arg{binnum}}	& \uline{\textbf{\%}[\textbf{-}]\arg{bindigit}\arg{bindigit}*} \\
\uline{\arg{cnum}}		& \uline{\textbf{'}\arg{char}\textbf{'}} \\
\uline{\arg{bindigit}}	& \uline{\{ \textbf{0} {\textbar} \textbf{1} \}} \\
\uline{\arg{decdigit}}	& \uline{\{
		\textbf{0} {\textbar} \textbf{1} {\textbar} \textbf{2} {\textbar}
		\textbf{3} {\textbar} \textbf{4} {\textbar}	\textbf{5} {\textbar}
		\textbf{6} {\textbar} \textbf{7} {\textbar} \textbf{8} {\textbar}
		\textbf{9} \}} \\
\uline{\arg{hexdigit}}	& \uline{\{ \arg{decdigit} {\textbar}
		\textbf{a} {\textbar} \textbf{b} {\textbar} \textbf{c} {\textbar}
		\textbf{d} {\textbar} \textbf{e} {\textbar} \textbf{f} {\textbar}
		\textbf{A} {\textbar} \textbf{B} {\textbar} \textbf{C} {\textbar}
		\textbf{D} {\textbar} \textbf{E} {\textbar} \textbf{F} \}}
\end{tabular}
\end{center}

\uline{\arg{bdigit} represents a digit according to the value of
\word{BASE} (see \xref{usage:digits}).
For \arg{hexdigit}, the digits \textbf{a}\ldots\textbf{f} have the
values 10\ldots15. \arg{char} represents any printable character.}

\uline{The radix used for number conversion is:}
\begin{center}
	\begin{tabular}{ll}
		\uline{\arg{BASEnum}}	& \uline{the value in \word{BASE}} \\
		\uline{\arg{decnum}}	& \uline{10} \\
		\uline{\arg{hexnum}}	& \uline{16} \\
		\uline{\arg{binnum}}	& \uline{2} \\
		\uline{\arg{cnum}}		& \uline{the number is the value of \arg{char}}
	\end{tabular}
\end{center}
\cbend


\subsection{Finding definition names} % 3.4.2
\label{usage:find}

A string matches a definition name if each character in the string
matches the corresponding character in the string used as the
definition name when the definition was created. The case
sensitivity (whether or not the upper-case letters match the
lower-case letters) is implementation defined. A system may be
either case sensitive, treating upper- and lower-case letters as
different and not matching, or case insensitive, ignoring
differences in case while searching.

The matching of upper- and lower-case letters with alphabetic
characters in character set extensions such as accented
international characters is implementation defined.

A system shall be capable of finding the definition names defined
by this Standard when they are spelled with upper-case letters.


\subsection{Semantics} % 3.4.3
\label{usage:semantics}

The semantics of a Forth definition are implemented by machine code
or a sequence of execution tokens or other representations. They are
largely specified by the stack notation in the glossary entries,
which shows what values shall be consumed and produced. The prose in
each glossary entry further specifies the definition's behavior.

Each Forth definition may have several behaviors, described in the
following sections. The terms ``initiation semantics'' and
``run-time semantics'' refer to definition fragments, and have
meaning only within the individual glossary entries where they
appear.

\subsubsection{Execution semantics} % 3.4.3.1

The execution semantics of each Forth definition are specified in an
``\textsf{Execution:}'' section of its glossary entry. When a
definition has only one specified behavior, the label is omitted.

Execution may occur implicitly, when the definition into which it
has been compiled is executed, or explicitly, when its execution
token is passed to \word{EXECUTE}. The execution semantics of a
syntactically correct definition under conditions other than those
specified in this Standard are implementation dependent.

Glossary entries for defining words include the execution semantics
for the new definition in a ``\emph{name} \textsf{Execution:}''
section.

\subsubsection{Interpretation semantics} % 3.4.3.2
\label{usage:interpret}

Unless otherwise specified in an ``\textsf{Interpretation:}''
section of the glossary entry, the interpretation semantics of a
Forth definition are its execution semantics.

A system shall be capable of executing, in interpretation state,
all of the definitions from the Core word set and any definitions
included from the optional word sets or word set extensions whose
interpretation semantics are defined by this Standard.

A system shall be capable of executing, in interpretation state,
any new definitions created in accordance with
\xref[Usage requirements]{usage}.

\subsubsection{Compilation semantics} % 3.4.3.3
\label{usage:compile}

Unless otherwise specified in a ``\textsf{Compilation:}'' section
of the glossary entry, the compilation semantics of a Forth
definition shall be to append its execution semantics to the
execution semantics of the current definition.


\subsection{Possible actions on an ambiguous condition} % 3.4.4
\label{usage:ambiguous}

When an ambiguous condition exists, a system may take one or more
of the following actions:

\begin{itemize}
\item ignore and continue;
\item display a message;
\item execute a particular word;
\item set interpretation state and begin text interpretation;
\item take other implementation-defined actions;
\item take implementation-dependent actions.
\end{itemize}

The response to a particular ambiguous condition need not be the
same under all circumstances.


\subsection{Compilation} % 3.4.5
\label{usage:compilation}

A program shall not attempt to nest compilation of definitions.

During the compilation of the current definition, a program shall
not execute any defining word, \word{:NONAME}, or any definition
that allocates dictionary data space. The compilation of the
current definition may be suspended using \word{[} (left-bracket)
and resumed using \word{]} (right-bracket). While the compilation
of the current definition is suspended, a program shall not execute
any defining word, \word{:NONAME}, or any definition that allocates
dictionary data space.
