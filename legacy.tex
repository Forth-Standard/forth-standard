\chapter{The optional Legacy word set} % x
\wordlist{legacy}
\cbstart

\section{Introduction} % x.1

This Standard adopts certain words and practices that cause
some previously used words and practices to become obsolescent.
Due to the widespread use of some of these words in legacy code,
they have been collected into this word set to preserve the
definition of these words.  These words should \emph{not} be
used in new developments.

\section{Additional terms and notation} % x.2

None.

\section{Additional usage requirements} % x.3

\subsection{Environmental queries} % x.3.1

Append table \ref{legacy:env} to table \ref{table:env}.

See: \xref[3.2.6 Environmental queries]{usage:env}.

\begin{table}[ht]
  \begin{center}
	\caption{Environmental Query Strings}
	\label{legacy:env}
	\begin{tabular}{p{9em}rcp{0.42\textwidth}}
		\hline\hline
		\multicolumn{2}{l}{String \hfill Value data type} & Constant? & Meaning \\
		\hline
		\texttt{LEGACY}		& \emph{flag}	& no	&
			legacy word set present \\
		\texttt{LEGACY-EXT}	& \emph{flag}	& no	&
			legacy extensions word set present \\
		\hline\hline
	\end{tabular}
  \end{center}
\end{table}

\subsection{Text interpreter input number conversion} % x.3.2

None.

\section{Additional documentation requirements} % x.4

\subsection{System documentation} % x.4.1

\subsubsection{Implementation-defined options} % x.4.1.1

\begin{itemize}
\item character editing of \wref{legacy:EXPECT}{EXPECT};
\item display after input terminates in \wref{legacy:EXPECT}{EXPECT};
\end{itemize}

\subsubsection{Ambiguous conditions} % x.4.1.2

\begin{itemize}
\item producing a result out of range, e.g., multiplication
	(using \word{*}) results in a value too big to be represented by
	a single-cell integer
	(\wref{legacy:CONVERT}{CONVERT};
\item deleting the compilation word-list (\wref{legacy:FORGET}{FORGET});
\item \emph{name} can't be found (\wref{legacy:FORGET}{FORGET});
\item removing a needed definition (\wref{legacy:FORGET}{FORGET}).
\end{itemize}

\subsubsection{Other system documentation} % x.4.1.3

\begin{itemize}
\item no additional requirements.
\end{itemize}

\subsection{Program documentation} % x.4.2

\subsubsection{Environmental dependencies} % x.4.2.1

\begin{itemize}
\item depending on the presence or absence of non-graphic characters
	in a received string (\wref{legacy:EXPECT}{EXPECT});
\end{itemize}


\section{Compliance and labeling} % x.5

\subsection{ANS Forth systems} % x.5.1

The phrase ``Providing the Legacy word set'' shall be appended to
the label of any Standard System that provides all of the Legacy
word set.

The phrase ``Providing \emph{name(s)} from the Legacy Extensions
word set'' shall be appended to the label of any Standard System
that provides portions of the Legacy Extensions word set.

The phrase ``Providing the Legacy Extensions word set'' shall be
appended to the label of any Standard System that provides all of
the Legacy and Legacy Extensions word sets.

\subsection{ANS Forth programs} % x.5.2

The phrase ``Requiring the Legacy word set'' shall be appended to
the label of Standard Programs that require the system to provide
the Legacy word set.

The phrase ``Requiring \emph{name(s)} from the Legacy Extensions
word set'' shall be appended to the label of Standard Programs that
require the system to provide portions of the Legacy Extensions
word set.

The phrase ``Requiring the Legacy Extensions word set'' shall be
appended to the label of Standard Programs that require the system
to provide all of the Legacy and Legacy Extensions word sets.
\cbend

\section{Glossary} % x.6

\subsection{Legacy words} % x.6.1


\begin{worddef}[numTIB]{0060}{\num{}TIB}[number-t-i-b]
\item \stack{}{a-addr}

	\param{a-addr} is the address of a cell containing the number
	of characters in the terminal input buffer.

\note \remove{x:legacy}{
	This word is obsolescent and is included as a concession to
	existing implementations.}

\place{x:legacy}{
	This function has been superseded by \word{SOURCE}.}

	\begin{rationale} % A.x.1.0060 #TIB
		The function of \word{numTIB} has been superseded by
		\wref{core:SOURCE}{SOURCE}.
	\end{rationale}
\end{worddef}


\begin{worddef}{0970}{CONVERT}
\item \stack{ud_1 c-addr_1}{ud_2 c-addr_2}

	\param{ud_2} is the result of converting the characters within
	the text beginning at the first character after \param{c-addr_1}
	into digits, using the number in \word{BASE}, and adding each
	digit to \param{ud_1} after multiplying \param{ud_1} by the
	number in \word{BASE}. Conversion continues until a character
	that is not convertible is encountered. \param{c-addr_2} is
	the location of the first unconverted character. An ambiguous
	condition exists if \param{ud_2} overflows.

\note \remove{x:legacy}{%
	This word is obsolescent and is included as a concession to
	existing implementations. Its}
	\place{x:legacy}{This} function is superseded by
	\wref{core:toNUMBER}{>NUMBER}.

\see \xref[3.2.1.2 Digit conversion]{usage:digits}.

	\begin{rationale} % A.6.2.0970 CONVERT
		\word{CONVERT} may be defined as follows:

		\tab \word{:} \word{CONVERT}~
			\word{CHAR+} \texttt{65535} \word{toNUMBER} \word{DROP}
		\word{;}
	\end{rationale}

	\begin{testing}\ttfamily
	\cbstart\patch{x:test}
		\textdf{Create two large integers, small enough to not
			cause overflow}

		MAX-INT  3 \word{/} \word{CONSTANT} cvi1 \\
		MAX-INT  5 \word{/} \word{CONSTANT} cvi2

		\textsf{Create a string of the form ``(n1digits.n2digits)''}

		\word{:} 2n>str  \word{p} +n1 +n2 -{}- caddr u ) \\
		\tab \word{num-start} \word{[CHAR]} ) \word{HOLD} \word{StoD} \word{numS} \word{2DROP}	\tab \word{p} -{}- +n1 ) \\
		\tab[2.3] \word{[CHAR]} . \word{HOLD} \word{StoD} \word{numS} \\
		\tab[2.3] \word{[CHAR]} ( \word{HOLD} \word{num-end}				\tab[6] \word{p} -{}- caddr1 u ) \\
		\tab \word{HERE} \word{SWAP} \word{2DUP} \word{2toR} \word{CHARS} \word{DUP} \word{ALLOT} \word{MOVE} \word{2Rfrom} \\
		\word{;}

		cvi1 cvi2 2n>str \word{CONSTANT} cv\$len \word{CONSTANT} cv\$ad

		\test{0 0 cv\$ad \word{CONVERT} \word{C@}}{cvi1 \word{StoD} \word{CHAR} .} \\
		\test{0 0 cv\$ad \word{CONVERT} \word{CONVERT} \word{C@}}{ \\
			\tab[1.2] 0 0 cv\$ad \word{CHAR+} cv\$len 1- \word{toNUMBER} \\
			\tab[1.2] \word{1-} \word{SWAP} \word{CHAR+} \word{SWAP} \word{toNUMBER} \word{2DROP} \word{CHAR} )}
	\cbend
	\end{testing}
\end{worddef}


\begin{worddef}{1390}{EXPECT}
\item \stack{c-addr +n}{}

	Receive a string of at most \param{+n} characters. Display graphic
	characters as they are received. A program that depends on
	the presence or absence of non-graphic characters in the
	string has an environmental dependency. The editing functions,
	if any, that the system performs in order to construct the
	string of characters are implementation-defined.

	Input terminates when an implementation-defined line terminator
	is received or when the string is \param{+n} characters long. When
	input terminates, nothing is appended to the string and the
	display is maintained in an implementation-defined way.

	Store the string at \param{c-addr} and its length in \word{SPAN}.

\note
	\remove{x:legacy}{%
	This word is obsolescent and is included as a concession to
	existing implementations. Its}
	\place{x:legacy}{This} function is superseded
	by \wref{core:ACCEPT}{ACCEPT}.

	\begin{rationale} % A.6.2.1390 EXPECT
		Specification of positive integer counts (\param{+n}) for
		\word{EXPECT} allows some implementors to continue their
		practice of using a zero or negative value as a flag to
		trigger special behavior. Insofar as such behavior is outside
		the Standard, Standard Programs cannot depend upon it, but
		the Technical Committee doesn't wish to preclude it
		unnecessarily. Since actual values are almost always small
		integers, no functionality is impaired by this restriction.
	\end{rationale}
\end{worddef}


\begin{worddef}{1580}{FORGET}
\item \stack{"<spaces>name"}{}

	Skip leading space delimiters. Parse \param{name} delimited by a
	space. Find \param{name}, then delete \param{name} from the
	dictionary along with all words added to the dictionary after
	\param{name}. An ambiguous condition exists if \param{name} cannot
	be found.

	If the Search-Order word set is present, \word{FORGET} searches
	the compilation word list. An ambiguous condition exists if the
	compilation word list is deleted.

	An ambiguous condition exists if \word{FORGET} removes a word
	required for correct execution.
	\remove{x:legacy}{Note: This word is obsolescent
	and is included as a concession to existing implementations.}

\see \xref[3.4.1 Parsing]{usage:parsing}.

	\begin{rationale} % A.15.6.2.1580 FORGET
		Typical use:
			{\ldots} \word{FORGET} \emph{name} {\ldots}

		\word{FORGET} assumes that all the information needed to
		restore the dictionary to its previous state is inferable
		somehow from the forgotten word. While this may be true in
		simple linear dictionary models, it is difficult to implement
		in other Forth systems; e.g., those with multiple address
		spaces. For example, if Forth is embedded in ROM, how does
		\word{FORGET} know how much RAM to recover when an array
		is forgotten? A general and preferred solution is provided by
		\word[core]{MARKER}.
	\end{rationale}
\end{worddef}


\begin{worddef}{2040}{QUERY}
\item \stack{}{}

	Make the user input device the input source. Receive input into
	the terminal input buffer, replacing any previous contents. Make
	the result, whose address is returned by \word{TIB}, the input
	buffer. Set \word{toIN} to zero.

\note
	\remove{x:legacy}{%
	This word is obsolescent and is included as a concession to
	existing implementations.}

	\begin{rationale} % A.6.2.2040 QUERY
		The function of \word{QUERY} may be performed with \word{ACCEPT}
		and \word{EVALUATE}.
	\end{rationale}
\end{worddef}


\begin{worddef}{2240}{SPAN}
\item \stack{}{a-addr}

	\param{a-addr} is the address of a cell containing the count of
	characters stored by the last execution of \word{EXPECT}.

\note \remove{x:legacy}{%
	This word is obsolescent and is included as a concession to
	existing implementations.}
\end{worddef}


\begin{worddef}{2290}{TIB}[t-i-b]
\item \stack{}{c-addr}

	\param{c-addr} is the address of the terminal input buffer.

\note \remove{x:legacy}{%
	This word is obsolescent and is included as a concession to
	existing implementations.}

	\place{x:legacy}{%
	This function has been superseded by \wref{core:SOURCE}{SOURCE}.}

	\begin{rationale} % A.6.2.2290 TIB
		The function of \word{TIB} has been superseded by
		\word{SOURCE}.
	\end{rationale}
\end{worddef}

\subsection{Legacy extension words} % x.x.2
\extended

None.

