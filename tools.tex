\chapter{The optional Programming-Tools word set} % 15
\wordlist{tools}

\begin{intro}
\section{The optional Programming-Tools word set} % A.15

These words have been in widespread common use since the earliest
Forth systems.

Although there are environmental dependencies intrinsic to programs
using an assembler, virtually all Forth systems provide such a
capability. Insofar as many Forth programs are intended for real-time
applications and are intrinsically non-portable for this reason, the
Technical Committee believes that providing a standard window into
assemblers is a useful contribution to Forth programmers.

Similarly, the programming aids \word{DUMP}, etc., are valuable tools
even though their specific formats will differ between CPUs and Forth
implementations. These words are primarily intended for use by the
programmer, and are rarely invoked in programs.

One of the original aims of Forth was to erase the boundary between
``user'' and ``programmer'' --- to give all possible power to anyone
who had occasion to use a computer. Nothing in the above labeling or
remarks should be construed to mean that this goal has been abandoned.
\end{intro}

\section{Introduction} % 15.1

This optional word set contains words most often used during the
development of applications.

\section{Additional terms and notation} % 15.2

None.

\section{Additional usage requirements} % 15.3

\subsection{Environmental queries} % 15.3.1

Append table \ref{tools:env} to table \ref{table:env}.

See: \xref[3.2.6 Environmental queries]{usage:env}.

\begin{table}[ht]
  \begin{center}
	\caption{Environmental Query Strings}
	\label{tools:env}
	\begin{tabular}{p{9em}rcp{0.42\textwidth}}
		\hline\hline
		\multicolumn{2}{l}{String \hfill Value data type} & Constant? & Meaning \\
		\hline
		\texttt{TOOLS}		& \emph{flag}		& no	&
			programming-tools word set present \\
		\texttt{TOOLS-EXT}	& \emph{flag}		& no	&
			programming-tools extensions word set present \\
%		\texttt{X:defined}	& --				& --	&
%			the \textsf{X:defined} extension is present \\
		\hline\hline
	\end{tabular}
  \end{center}
\end{table}

\subsection{The Forth dictionary} % 15.3.2
\label{tools:dict}

A program using the words \word{CODE} or \word{;CODE} associated
with assembler code has an environmental dependency on that
particular instruction set and assembler notation.

Programs using the words \word{EDITOR} or \word{ASSEMBLER} require
the Search Order word set or an equivalent implementation-defined
capability.

See: \xref[3.3 The Forth dictionary]{usage:dict}.

\section{Additional documentation requirements} % 15.4

\subsection{System documentation} % 15.4.1

\subsubsection{Implementation-defined options} % 15.4.1.1

\begin{itemize}
\item ending sequence for input following
	\wref{tools:;CODE}{;CODE} and
	\wref{tools:CODE}{CODE};
\item manner of processing input following
	\wref{tools:;CODE}{;CODE} and
	\wref{tools:CODE}{CODE};
\item search-order capability for
	\wref{tools:EDITOR}{EDITOR} and
	\wref{tools:ASSEMBLER}{ASSEMBLER}
	(\xref[15.3.3 The Forth dictionary]{tools:dict});
\item source and format of display by \wref{tools:SEE}{SEE}.
\end{itemize}

\subsubsection{Ambiguous conditions} % 15.4.1.2

\begin{itemize}
\item deleting the compilation word-list (\wref{tools:FORGET}{FORGET});
\item fewer than $u+1$ items on control-flow stack
	(\wref{tools:CS-PICK}{CS-PICK},
	 \wref{tools:CS-ROLL}{CS-ROLL});
\item \emph{name} can't be found (\wref{tools:FORGET}{FORGET});
\item \emph{name} not defined via \wref{core:CREATE}{CREATE}
	(\wref{tools:;CODE}{;CODE});
\item \wref{core:POSTPONE}{POSTPONE} applied to \wref{tools:[IF]}{[IF]};
\item reaching the end of the input source before matching
	\wref{tools:[ELSE]}{[ELSE]} or \wref{tools:[THEN]}{[THEN]}
	(\wref{tools:[IF]}{[IF]});
\item removing a needed definition (\wref{tools:FORGET}{FORGET}).
\end{itemize}

\subsubsection{Other system documentation} % 15.4.1.3

\begin{itemize}
\item no additional requirements.
\end{itemize}

\subsection{Program documentation} % 15.4.2

\subsubsection{Environmental dependencies} % 15.4.2.1

\begin{itemize}
\item using the words \wref{tools:;CODE}{;CODE} or
	\wref{tools:CODE}{CODE}.
\end{itemize}

\subsubsection{Other program documentation} % 15.4.2.2

\begin{itemize}
\item no additional requirements.
\end{itemize}

\section{Compliance and labeling} % 15.5

\subsection{ANS Forth systems} % 15.5.1

The phrase ``Providing the Programming-Tools word set'' shall be
appended to the label of any Standard System that provides all of
the Programming-Tools word set.

The phrase ``Providing \emph{name(s)} from the Programming-Tools
Extensions word set'' shall be appended to the label of any Standard
System that provides portions of the Programming-Tools Extensions
word set.

The phrase ``Providing the Programming-Tools Extensions word set''
shall be appended to the label of any Standard System that provides
all of the Programming-Tools and Programming-Tools Extensions word
sets.

\subsection{ANS Forth programs} % 15.5.2

The phrase ``Requiring the Programming-Tools word set'' shall be
appended to the label of Standard Programs that require the system
to provide the Programming-Tools word set.

The phrase ``Requiring \emph{name(s)} from the Programming-Tools
Extensions word set'' shall be appended to the label of Standard
Programs that require the system to provide portions of the
Programming-Tools Extensions word set.

The phrase ``Requiring the Programming-Tools Extensions word set''
shall be appended to the label of Standard Programs that require the
system to provide all of the Programming-Tools and Programming-Tools
Extensions word sets.



\section{Glossary} % 15.6

\begin{intro}
\subsection{Glossary} % A.15.6
\end{intro}


\subsection{Programming-Tools words} % 15.6.1

\begin{worddef}{0220}{.S}[dot-s]
\item \stack{}{}

	Copy and display the values currently on the data stack. The
	format of the display is implementation-dependent.

	\word{.S} may be implemented using pictured numeric output words.
	Consequently, its use may corrupt the transient region identified
	by \word[core]{num-end}.

\see \xref[3.3.3.6 Other transient regions]{usage:transient}.

	\begin{defer}
	\rationale % A.15.6.1.0220 .S
		\word[tools]{.S} is a debugging convenience found on almost
		all Forth systems. It is universally mentioned in Forth texts.
	\end{defer}
\end{worddef}


\begin{worddef}[q]{0600}{?}[question]
\item \stack{a-addr}{}

	Display the value stored at \param{a-addr}.

	\word{q} may be implemented using pictured numeric output words.
	Consequently, its use may corrupt the transient region identified
	by \word[core]{num-end}.

\see \xref[3.3.3.6 Other transient regions]{usage:transient}.
\end{worddef}


\begin{worddef}{1280}{DUMP}
\item \stack{addr u}{}

	Display the contents of \param{u} consecutive addresses starting
	at \param{addr}. The format of the display is implementation
	dependent.

	\word{DUMP} may be implemented using pictured numeric output
	words. Consequently, its use may corrupt the transient region
	identified by \word[core]{num-end}.

\see \xref[3.3.3.6 Other transient regions]{usage:transient}.
\end{worddef}


\begin{worddef}{2194}{SEE}
\item \stack{"<spaces>name"}{}

	Display a human-readable representation of the named word's
	definition. The source of the representation (object-code
	decompilation, source block, etc.) and the particular form of
	the display is implementation defined.

	\word{SEE} may be implemented using pictured numeric output
	words. Consequently, its use may corrupt the transient region
	identified by \word[core]{num-end}.

\see \xref[3.3.3.6 Other transient regions]{usage:transient}.

	\begin{defer}
	\rationale % A.15.6.1.2194 SEE
		\word[tools]{SEE} acts as an on-line form of documentation of
		words, allowing modification of words by decompiling and
		regenerating with appropriate changes.
	\end{defer}
\end{worddef}


\begin{worddef}{2465}{WORDS}
\item \stack{}{}

	List the definition names in the first word list of the search
	order. The format of the display is implementation-dependent.

	\word{WORDS} may be implemented using pictured numeric output
	words. Consequently, its use may corrupt the transient region
	identified by \word[core]{num-end}.

\see \xref[3.3.3.6 Other transient regions]{usage:transient}.

	\begin{defer}
	\rationale % A.15.6.1.2465 WORDS
		\word[tools]{WORDS} is a debugging convenience found on almost
		all Forth systems. It is universally referred to in Forth texts.
	\end{defer}
\end{worddef}


\subsection{Programming-Tools extension words} % 15.6.2
\extended

\begin{worddef}{0470}{;CODE}[semicolon-code]
\interpret
	Interpretation semantics for this word are undefined.

\compile
	\stack[C]{colon-sys}{}

	Append the run-time semantics below to the current definition.
	End the current definition, allow it to be found in the
	dictionary, and enter interpretation state, consuming
	\param{colon-sys}.

	Subsequent characters in the parse area typically represent
	source code in a programming language, usually some form of
	assembly language. Those characters are processed in an
	implementation-defined manner, generating the corresponding
	machine code. The process continues, refilling the input buffer
	as needed, until an implementation-defined ending sequence is
	processed.

\runtime
	\stack{}{}
	\stack[R]{nest-sys}{}

	Replace the execution semantics of the most recent definition
	with the \emph{name} execution semantics given below. Return
	control to the calling definition specified by \param{nest-sys}.
	An ambiguous condition exists if the most recent definition was
	not defined with \word[core]{CREATE} or a user-defined word that
	calls \word[core]{CREATE}.

\execute[name]
	\stack{i*x}{j*x}

	Perform the machine code sequence that was generated following
	\word{;CODE}.

\see \wref{core:DOES}{DOES>}.

	\begin{defer}
	\rationale % A.15.6.2.0470 ;CODE
		Typical use:
		\word[core]{:} \texttt{namex}
			{\ldots} \arg{create} {\ldots}
		\word{;CODE} {\ldots}

		where \texttt{namex} is a defining word, and \arg{create} is
		\word[core]{CREATE} or any user defined word that calls
		\word[core]{CREATE}.
	\end{defer}
\end{worddef}


\begin{worddef}{0702}{AHEAD}
\interpret
	Interpretation semantics for this word are undefined.

\compile
	\stack[C]{}{orig}

	Put the location of a new unresolved forward reference
	\param{orig} onto the control flow stack. Append the run-time
	semantics given below to the current definition. The semantics
	are incomplete until \param{orig} is resolved (e.g., by
	\word[core]{THEN}).

\runtime
	\stack{}{}

	Continue execution at the location specified by the resolution
	of \param{orig}.
\end{worddef}


\begin{worddef}{0740}{ASSEMBLER}
\item \stack{}{}

	Replace the first word list in the search order with the
	\word{ASSEMBLER} word list.

\see \xref[16. The optional Search-Order word set]{wordlist:search}.
\end{worddef}


\begin{worddef}{0830}{BYE}
\item \stack{}{}

	Return control to the host operating system, if any.
\end{worddef}


\begin{worddef}{0930}{CODE}
\item \stack{"<spaces>name"}{}

	Skip leading space delimiters. Parse \param{name} delimited by a
	space. Create a definition for \param{name}, called a ``code
	definition'', with the execution semantics defined below.

	Subsequent characters in the parse area typically represent
	source code in a programming language, usually some form of
	assembly language. Those characters are processed in an
	implementation-defined manner, generating the corresponding
	machine code. The process continues, refilling the input buffer
	as needed, until an implementation-defined ending sequence is
	processed.

\execute[name]
	\stack{i*x}{j*x}

	Execute the machine code sequence that was generated following
	\word{CODE}.

\see \xref[3.4.1 Parsing]{usage:parsing}.

	\begin{defer}
	\rationale % A.15.6.2.0930 CODE
		Some Forth systems implement the assembly function by adding
		an \word[tools]{ASSEMBLER} word list to the search order,
		using the text interpreter to parse a postfix assembly
		language with lexical characteristics similar to Forth source
		code. Typically, in such systems, assembly ends when a word
		\texttt{END-CODE} is interpreted.
	\end{defer}
\end{worddef}


\begin{worddef}{1015}{CS-PICK}[c-s-pick]
\interpret
	Interpretation semantics for this word are undefined.

\execute
	\stack[C]{dest_u {\ldots} orig_0|dest_0}{dest_u {\ldots} orig_0|dest_0 dest_u}
	\stack[S]{u}{}

	Remove \param{u}. Copy \param{dest_u} to the top of the
	control-flow stack. An ambiguous condition exists if there
	are less than \param{u}+1 items, each of which shall be an
	\param{orig} or \param{dest}, on the control-flow stack
	before \word{CS-PICK} is executed.

	If the control-flow stack is implemented using the data stack,
	\param{u} shall be the topmost item on the data stack.

	\begin{defer}
	\rationale % A.15.6.2.1015 CS-PICK
		The intent is to reiterate a \param{dest} on the control-flow
		stack so that it can be resolved more than once. For example:
		\setwordlist{core}
		\begin{quote}\ttfamily
			\word{bs} Conditionally transfer control to beginning of loop \\
			\word{bs} This is similar in spirit to C's "continue" statement.

			\word{:} ?REPEAT ~~\word{p} dest -- dest ) \word{bs} Compilation \\
			\tab\tab\tab~\word{p} flag -- ) \tab~\word{bs} Execution \\
			\tab 0 \word[tools]{CS-PICK} ~ \word{POSTPONE} \word{UNTIL} \\
			\word{;} \word{IMMEDIATE}

			\word{:} XX \word{p} -- ) \word{bs} Example use of ?REPEAT \\
			\tab \word{BEGIN} \\
			\tab\tab {\ldots} \\
			\tab flag ?REPEAT \word{p} Go back to BEGIN if flag is false ) \\
			\tab\tab {\ldots} \\
			\tab flag ?REPEAT \word{p} Go back to BEGIN if flag is false ) \\
			\tab\tab {\ldots} \\
			\tab flag \word{UNTIL}~~ \word{p} Go back to BEGIN if flag is false ) \\
			\word{;}
		\end{quote}
		\setwordlist{tools}
	\end{defer}
\end{worddef}


\begin{worddef}{1020}{CS-ROLL}[c-s-roll]
\interpret
	Interpretation semantics for this word are undefined.

\execute
	\stack[C]{orig_u|dest_u orig_{u-1}|dest_{u-1} {\ldots} orig_0|dest_0}
			 {orig_{u-1}|dest_{u-1} {\ldots} orig_0|dest_0 orig_u|dest_u}
	\stack[S]{u}{}

	Remove \param{u}. Rotate \param{u}+1 elements on top of the
	control-flow stack so that \param{orig_u|dest_u} is on top of
	the control-flow stack. An ambiguous condition exists if there
	are less than \param{u}+1 items, each of which shall be an
	\param{orig} or \param{dest}, on the control-flow stack before
	\word{CS-ROLL} is executed.

	If the control-flow stack is implemented using the data stack,
	\param{u} shall be the topmost item on the data stack.

	\begin{defer}
	\rationale % A.15.6.2.1020 CS-ROLL
		The intent is to modify the order in which the \param{orig}s
		and \param{dest}s on the control-flow stack are to be resolved
		by subsequent control-flow words. For example, \word[core]{WHILE}
		could be implemented in terms of \word[core]{IF} and
		\word{CS-ROLL}, as follows:

		\setwordlist{core}
		\begin{quote}\ttfamily
			\word{:} \word{WHILE} ~ \word{p} dest -- orig dest ) \\
			\tab \word{POSTPONE} \word{IF} ~ 1 \word[tools]{CS-ROLL} \\
			\word{;} \word{IMMEDIATE}
		\end{quote}
		\setwordlist{tools}
	\end{defer}
\end{worddef}


\begin{worddef}{1300}{EDITOR}
\item \stack{}{}

	Replace the first word list in the search order with the
	\word{EDITOR} word list.

\see \xref[16. The Optional Search-Order Word Set]{wordlist:search}.
\end{worddef}


\begin{worddef}{1580}{FORGET}
\item \stack{"<spaces>name"}{}

	Skip leading space delimiters. Parse \param{name} delimited by a
	space. Find \param{name}, then delete \param{name} from the
	dictionary along with all words added to the dictionary after
	\param{name}. An ambiguous condition exists if \param{name} cannot
	be found.

	If the Search-Order word set is present, \word{FORGET} searches
	the compilation word list. An ambiguous condition exists if the
	compilation word list is deleted.

	An ambiguous condition exists if \word{FORGET} removes a word
	required for correct execution. Note: This word is obsolescent
	and is included as a concession to existing implementations.

\see \xref[3.4.1 Parsing]{usage:parsing}.

	\begin{defer}
	\rationale % A.15.6.2.1580 FORGET
		Typical use:
			{\ldots} \word{FORGET} \emph{name} {\ldots}

		\word{FORGET} assumes that all the information needed to
		restore the dictionary to its previous state is inferable
		somehow from the forgotten word. While this may be true in
		simple linear dictionary models, it is difficult to implement
		in other Forth systems; e.g., those with multiple address
		spaces. For example, if Forth is embedded in ROM, how does
		\word{FORGET} know how much RAM to recover when an array
		is forgotten? A general and preferred solution is provided by
		\word[core]{MARKER}.
	\end{defer}
\end{worddef}


\begin{worddef}{2250}{STATE}
\item \stack{}{a-addr}

	Extend the semantics of \wref{core:STATE}{STATE} to allow
	\word{;CODE} to change the value in \word{STATE}. A program
	shall not directly alter the contents of \word{STATE}.

\see \xref[3.4 The Forth text interpreter]{usage:command},
	\wref{core::}{:},
	\wref{core:;}{;},
	\wref{core:ABORT}{ABORT},
	\wref{core:QUIT}{QUIT},
	\wref{core:STATE}{STATE},
	\wref{core:[}{[},
	\wref{core:]}{]},
	\wref{core::NONAME}{:NONAME},
	\wref{tools:;CODE}{;CODE}.
\end{worddef}

\begin{worddef}{}{[DEFINED]}[bracket-defined][X:defined]
\compile
	Perform the execution semantics given below.

\execute
	\stack{"<spaces>name {\ldots}"}{flag}

	Skip leading space delimiters.  Parse name delimited by a space.
	Return a true flag if \param{name} is the name of a word that can
	be found (according to the rules in the system's \word{FIND});
	otherwise return a false flag.
	\word{[DEFINED]} is an immediate word.

	\begin{defer}
	\implementation
		\word{:} \word{[DEFINED]} \word{BL} \word{WORD} \word{FIND} \word{NIP} \word{0ne} \word{;} \word{IMMEDIATE}
	\end{defer}
\end{worddef}


\begin{worddef}{2531}{[ELSE]}[bracket-else]
\compile
	Perform the execution semantics given below.

\execute
	\stack{"<spaces>name {\ldots}"}{}

	Skipping leading spaces, parse and discard space-delimited words
	from the parse area, including nested occurrences of \word{[IF]}
	{\ldots} \word{[THEN]} and \word{[IF]} {\ldots} \word{[ELSE]}
	{\ldots} \word{[THEN]}, until the word \word{[THEN]} has been
	parsed and discarded. If the parse area becomes exhausted, it is
	refilled as with \word[core]{REFILL}. \word{[ELSE]} is an
	immediate word.

\see \xref[3.4.1 Parsing]{usage:parsing}.

	\begin{defer}
	\rationale % A.15.6.2.2531 [ELSE]
		Typical use:
			{\ldots} \emph{flag}
			\word[tools]{[IF]} {\ldots}
			\word[tools]{[ELSE]} {\ldots}
			\word[tools]{[THEN]} {\ldots}
	\end{defer}
\end{worddef}


\begin{worddef}{2532}{[IF]}[bracket-if]
\compile
	Perform the execution semantics given below.

\execute
	\stack{flag|flag "<spaces>name {\ldots}"}{}

	If \param{flag} is true, do nothing. Otherwise, skipping leading
	spaces, parse and discard space-delimited words from the parse
	area, including nested occurrences of \word{[IF]} {\ldots}
	\word{[THEN]} and \word{[IF]} {\ldots} \word{[ELSE]} {\ldots}
	\word{[THEN]}, until either the word \word{[ELSE]} or the word
	\word{[THEN]} has been parsed and discarded. If the parse area
	becomes exhausted, it is refilled as with \word[core]{REFILL}.
	\word{[IF]} is an immediate word.

	An ambiguous condition exists if \word{[IF]} is
	\word[core]{POSTPONE}d, or if the end of the input buffer is
	reached and cannot be refilled before the terminating
	\word{[ELSE]} or \word{[THEN]} is parsed.

\see \xref[3.4.1 Parsing]{usage:parsing}.

	\begin{defer}
	\rationale % A.15.6.2.2532 [IF]
		Typical use:
			{\ldots} \emph{flag}
			\word[tools]{[IF]} {\ldots}
			\word[tools]{[ELSE]} {\ldots}
			\word[tools]{[THEN]} {\ldots}
	\end{defer}
\end{worddef}


\begin{worddef}{2533}{[THEN]}[bracket-then]
\compile
	Perform the execution semantics given below.

\execute
	\stack{}{}

	Does nothing. \word{[THEN]} is an immediate word.

	\begin{defer}
	\rationale % A.15.6.2.2533 [THEN]
		Typical use:
			{\ldots} \emph{flag}
			\word[tools]{[IF]} {\ldots}
			\word[tools]{[ELSE]} {\ldots}
			\word[tools]{[THEN]} {\ldots}

		Software that runs in several system environments often
		contains some source code that is environmentally dependent.
		Conditional compilation --- the selective inclusion or
		exclusion of portions of the source code at compile time ---
		is one technique that is often used to assist in the
		maintenance of such source code.

		Conditional compilation is sometimes done with ``smart
		comments'' --- definitions that either skip or do not skip
		the remainder of the line based on some test. For example:

		\setwordlist{core}
		\begin{quote}\ttfamily
			\word{bs} If 16-Bit? contains TRUE, lines preceded by 16BIT\bs \\
			\word{bs} will be skipped. Otherwise, they will not be skipped.

			\word{VARIABLE} 16-BIT?

			\word{:} 16BIT\bs~~\word{p} -- ) ~
				16-BIT? \word{@} ~
				\word{IF} ~ \word{POSTPONE} \word{bs} ~ \word{THEN} \\
			\word{;} \word{IMMEDIATE}
		\end{quote}
		\setwordlist{tools}

		This technique works on a line by line basis, and is good for
		short, isolated variant code sequences.

		More complicated conditional compilation problems suggest a
		nestable method that can encompass more than one source line
		at a time. The words included in the ANS Forth optional
		Programming tools extensions word set are useful for this
		purpose. The implementation given below works with any input
		source (keyboard, \word{EVALUATE}, \word[block]{BLOCK}, or
		text file).

		\begin{quote}\ttfamily
		\begin{tabbing}
		  \tab \= \tab \= \tab \= \tab \= \tab \= \tab \= \hspace*{17em} \= \kill
		  \+
		  \word{:} \word{[ELSE]} \word{p} -- ) \\
			\+ 1 \word{BEGIN}																		\>\>\>\>\>\>\word{bs}~level \\
				\+ \word{BEGIN} ~ \word{BL} \word{WORD} \word{COUNT} ~ \word{DUP} ~ \word{WHILE}	\>\>\>\>\>	\word{bs}~level adr len \\
					\word{2DUP} \word{Sq} [IF]"  \word[string]{COMPARE} \word{0=} \word{IF}			\>\>\>\>	\word{bs}~level adr len \\
					\> \word{2DROP} \word{1+}														\>\>\>		\word{bs}~level' \\
					\+ \word{ELSE}																	\>\>\>\>	\word{bs}~level adr len \\
						\word{2DUP} \word{Sq} [ELSE]" \word[string]{COMPARE} \word{0=} \word{IF}	\>\>\>		\word{bs}~level adr len \\
						\> \word{2DROP} \word{1-} \word{DUP} \word{IF} \word{1+} \word{THEN}		\>\>		\word{bs}~level' \\
						\word{ELSE}																	\>\>\>		\word{bs}~level adr len \\
						\> \word{Sq} [THEN]" \word[string]{COMPARE} \word{0=} \word{IF}				\>\>		\word{bs}~level \\
						\>\> \word{1-}																\>			\word{bs}~level' \\
						\>\word{THEN} \\
						\- \word{THEN} \\
					\- \word{THEN} \word{qDUP} \word{0=} ~ \word{IF} \word{EXIT} \word{THEN}		\>\>\>\>	\word{bs}~level' \\
				\- \word{REPEAT} \word{2DROP}														\>\>\>\>\>	\word{bs}~level \\
			\word{REFILL} \word{0=} \word{UNTIL}													\>\>\>\>\>\>\word{bs}~level \\
			\- \word{DROP} \\
		\word{;} \word{IMMEDIATE} \\[1.5\parskip]

		\word{:} \word[tools]{[IF]} \word{p} flag -- ) \\
		\> \word{0=} \word{IF} \word{POSTPONE} \word[tools]{[ELSE]} \word{THEN} \\
		\word{;} \word{IMMEDIATE} \\[1.5\parskip]

		\word{:} \word{[THEN]} \word{p} -- ) \word{;} \word{IMMEDIATE} \\
		\end{tabbing}
		\end{quote}
	\end{defer}
\end{worddef}


\begin{worddef}{}{[UNDEFINED]}[bracket-undefined][X:defined]
\compile
	Perform the execution semantics given below.

\execute
	\stack{"<spaces>name {\ldots}"}{flag}

	Skip leading space delimiters.  Parse name delimited by a space.
	Return a false flag if \param{name} is the name of a word that
	can be found (according to the rules in the system's \word{FIND});
	otherwise return a true flag. \word{[UNDEFINED]} is an immediate
	word.

	\begin{defer}
	\implementation
		\word{:} \word{[UNDEFINED]} \word{BL} \word{WORD} \word{FIND} \word{NIP} \word{0=} \word{;} \word{IMMEDIATE}
	\end{defer}
\end{worddef}
